\section{Neutrino Interactions}
\label{sec:Interactions}

\noindent{\sc Models of the Nucleus and Neutrino Nucleus Event Generators}
\\The neutrino has shown itself to be a much more complicated particle than expected; it has flavor states composed of an oscillating mixture of mass eigenstates. In the United States, significant investment is being made to further investigate this profound behavior of the neutrino.  In particular, our sights are set on precisely testing the current 3-neutrino mixing picture and determining whether or not neutrinos violate CP, the latter of which can best be tested with accelerator-based sources of neutrinos. In accelerator-based neutrino oscillation experiments, the goal is to measure the appearance and/or disappearance of a given neutrino flavor as a function of the incident neutrino energy.  The signal for appearance or disappearance can be the observation of an exclusive process or inclusive production of a particular neutrino flavor in the detector.  Since heavy-nucleus-based detectors are necessary for a large detector mass, essential to creating the event rates required for sensitive measurements, the neutrino must interact with a complex nucleus.  Obviously then, neutrino nucleus interactions necessarily play a central role in accelerator based neutrino oscillation experiments. 

Although measurements and theoretical work are needed to characterize neutrino interactions in the low energy regime ($E_\nu  \leq$ 100~MeV), especially relevant for core-collapse supernova neutrinos and development of future underground detectors,
the main challenge that must be confronted is with massive detectors viewing energetic neutrinos ($E_\nu  \geq$  500~MeV) interacting with atomic nuclei.  Particularly now that the neutrino oscillation experiments are evolving from the discovery to the precision stage, understanding the challenging role of the nucleus in neutrino interactions has become essential to help classify signal from background and minimize systematic uncertainties in our neutrino oscillation investigations (For details see NSAC Position Paper "Nuclear Theory and the U.S. Experimental Neutrino Physics Program", Alvarez-Ruso, et al.).    
 
The experience gathered during decades of hadronic and nuclear physics research has allowed the development of more precise and complete neutrino interaction models and has provided a more realistic description of the nuclear ground state and final state interactions. However, current event generators employ models of the nucleus that are far too simple and their limitations readily lead to extensions based on wrong physics. At present there exist much better and well-tested nuclear models than the relativistic Fermi gas that can be provided by nuclear theorists.  In collaborations between neutrino experimentalists and nuclear theorists, these more advanced models need to be implemented consistently in the detector event generators employed in the analysis of experimental results. 
A clear example of this synergy has emerged in connection with the CCQE measurements on nuclear targets performed by the MiniBooNE experiment~\cite{AguilarArevalo:2010zc}. Only after taking into account two-nucleon mechanisms (two-particle-two-hole excitations) has it been possible to reconcile these experimental results with the, admittedly limited, information on the nucleon axial form-factor available from neutrino scattering on deuterium and pion electroproduction~\cite{Amaro:2010sd,Nieves:2011yp,Martini:2011wp}.  These findings have important implications for neutrino-oscillation experiments as a source of systematic error in the determination of the neutrino energy, which is not known for the non-monochromatic neutrino beams.  

The development of neutrino detectors with high-resolution capability such as fine-grained sampling detectors and liquid argon has called special attention to dealing with so called final state interactions (FSI). While the theory discussed above is capable of calculating inclusive cross sections~\cite{Lovato:2014eva,1412.3081,Lovato:2015qka}, calculating the variety of exclusive final states that can be observed exceeds any current capability. Extensive recourse to phenomenology will be required and most of the necessary input will likely come from electron scattering carried out in kinematic regimes similar to those encountered in the neutrino beams. In that regard the electron beam at the Jefferson Laboratory is ideal and there exists a recent Jlab PAC approved proposal to carry out an (e,e$^\prime$p) measurement on $^{40}$Ar that will be very informative. 
Furthermore, in order to be most useful to the HEP oscillation program the current neutrino nucleus theoretical developments must be extended to higher-A nuclei such as Ar, to more relativistic energies/momenta and to include resonance production cross sections.  {\bf To most efficiently accomplish this ambitious program will require additional financial support of nuclear physics theorists working in this area.} 

Significantly, this important work by the nuclear theorists has to be packaged in a form that can be swiftly incorporated in neutrino event generators and it is not yet clear as to the best way to do this. It will likely involve extensive collaboration between nuclear theorists, the builders of event generators and neutrino experimentalists.   {\bf It should be obvious that the critical role of neutrino nucleus event generators needs to be emphasized and more community resources devoted to keeping them widely available, accurate, transparent, and current.}

\noindent{\sc The Present and Future Neutrino Interaction Experimental Program}
\\It is critical to benchmark the generators against both accelerator-based neutrino-nucleus interaction measurements and, via a collaborative HEP and NP effort, electron-nucleus interaction measurements.  Recent years have witnessed intense experimental activity aimed at a better understanding of neutrino interactions with nucleons and nuclei. A wealth of data exists through measurements of CC and NC processes made by ArgoNeuT~\cite{Acciarri:2014}, MINER$\nu$A~\cite{MINERvA}, MiniBooNE~\cite{miniboone}, MINOS near detector~\cite{minos}, NOMAD, SciBooNE~\cite{sciboone}, and T2K near detectors~\cite{Abe:2014nox}. In addition, the MicroBooNE experiment and NO$\nu$A near detector are beginning operation.  {\bf The current experimental neutrino interaction program (MINER$\nu$A, NO$\nu$A-ND, MicroBooNE, T2K Near Detector) continues to provide important data and should be supported to its conclusion. This includes efforts to improve the precision with which the neutrino flux is known.}  
 
To help refine the nuclear model in event generators, future neutrino interaction measurements that span a range of neutrino and antineutrino beam energies as well as target materials will be needed to extend the current program of GeV-scale neutrino interactions. {\bf Such new experiments have been proposed (CAPTAIN-MINER$\nu$A, LAr1-ND~\cite{Adams:2013uaa}, nuPRISM~\cite{Bhadra:2014oma}) and, if approved, should be supported to conclusion. Current and future long-and-short- baseline neutrino oscillation programs should evaluate what additional neutrino nucleus interaction data is required to meet their ambitious goals and support experiments that provide this data.} 

Although this activity has been stimulated mostly by the needs of neutrino oscillation experiments in their quest for a precise determination of neutrino properties, the relevance of neutrino interactions with matter extends over a large variety of topics, including hadronic and nuclear physics.  Neutrino cross section measurements permit the investigation of the axial structure of the nucleon and baryon resonances, enlarging our views of hadron structure beyond what is presently known from experiments with hadronic and electromagnetic probes and lattice QCD. In the recent past, the electromagnetic form factors of the nucleon have been extensively studied at JLab with unexpected results such as the differing dependence of the electric and magnetic proton form factors over the $Q^2$ range from 1 to 8.5~GeV$^2$ and their deviation from the approximate dipole behavior at $Q^2 > 1$~GeV$^2$~\cite{Perdrisat:2006hj}. In contrast, the experimental information about the axial nucleon form factors is scarce. The existing data from neutrino quasi-elastic scattering on deuterium and pion electroproduction are compatible with a dipole behavior which is not well justified from a theoretical point of view~\cite{Bhattacharya:2011ah}. {\bf This points out the need for a feasibility study of a high-statistics deuterium experiment.  We are still dependent on low-statistics deuterium bubble chamber results from over 30 years ago.} 

Understanding the excitation spectrum of the nucleon, and the properties of baryon resonances in general,  is a central question in strong interaction physics. Our knowledge about this spectrum was originally provided by elastic pion-nucleon scattering however recent experiments at MAMI, ELSA and JLab with photons and electrons have also unraveled the electromagnetic properties of baryon resonances. In contrast, the axial sector is practically unknown. Progress in this direction requires new and more precise measurements of neutrino inelastic scattering on hydrogen and deuterium targets.  {\bf This again stress the need for a new high-statistics experiment using a deuterium and hydrogen targets.}

For nuclear physics, the fact that neutrino experiments are performed with nuclear targets represents a challenge but also an opportunity as neutrino-nucleus interactions incorporate new and important information due to the presence of both  axial and vector currents. They provide an excellent testing ground for models of the axial response. It is interesting to elucidate the role of multi-nucleon mechanisms, in particular meson exchange currents which are different than in the case of electromagnetic probes; the same is true for long-range correlations because the axial current gets renormalized in the nuclear medium in a different way than the vector current. Modern experiments, most notably MINER$\nu$A~\cite{Tice:2014pgu}, will measure different nuclear targets with the same (anti)neutrino flux, and the suite of current and proposed experiments include light (CH,H$_2$O) and heavy nuclei (Ar, Fe and Pb), which will provide important clues about the mass dependence of different observables. Novel approaches are also being investigated experimentally, such as the generation of pseudo-monochromatic beams on a single target material akin to electron-scattering (nuPRISM detector~\cite{Bhadra:2014oma}).

\noindent{\sc Conclusion}
\\Understanding the subtleties of the nuclear environment and its effects on what neutrino experimentalists measure in their detectors can only be accurately performed with the input of nuclear physics theorists specializing in this topic. It is important to have an established  procedure that promotes nuclear theorists to join neutrino generator experts and neutrino experimentalists in working toward this goal. NuSTEC (Neutrino Scattering Theorist Experimentalist collaboration, {\tt http://nustec2014.phys.vt.edu}) is a collaboration established directly to provide this environment.  NuSTEC is encouraging the participation of individual nuclear theorists and neutrino experimentalists to explore neutrino-nucleus scattering with the goal of producing improved models that can be employed in neutrino-nucleus scattering event generators such as GENIE.  A current example of such a project is the work on Green's Function Monte Carlo techniques by a collaboration of Argonne, Jefferson Lab, and Los Alamos nuclear theorists with Fermilab neutrino experimentalists.  A white paper (Lovato, A. et al "Quantum Monte Carlo Methods for Neutrino-nucleus Interactions", FERMILAB-FN-0997-ND-T, JLAB-THY-15-2012, LA-UR-15-21054) has been submitted to the DOE-HEP and DOE-NP and is now being expanded to a full proposal to be submitted to a possible HEP-NP FOA.
 
