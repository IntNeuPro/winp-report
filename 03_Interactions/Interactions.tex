\section{Neutrino Interactions}
\label{sec:Interactions}

Significant investment is being made using
accelerator-based sources of neutrinos to critically test the current
3-neutrino mixing picture and determine whether or not neutrinos
violate CP.
%In accelerator-based neutrino oscillation experiments, the goal is to measure the appearance and/or disappearance of a given neutrino flavor as a function of the incident neutrino energy. 
The detectors associated with this program use heavy nuclei to achieve
the required event rates and modeling the interaction of neutrinos
with these heavy nuclei is required.  Now that neutrino
oscillation experiments are evolving from the discovery to the
precision stage, understanding the challenging role of the nucleus in
neutrino interactions has become an imperative. It is needed to sort
signal from background and to identify and minimize systematic
uncertainties in neutrino oscillation investigations.
%Obviously then, neutrino nucleus interactions necessarily play a central role in accelerator based neutrino oscillation experiments.  
It has therefore become essential to establish a more robust program
of characterizing neutrino-nucleus interactions and promptly
integrating these results into event generators. This difficult task
requires the cooperative effort of theory and experiment from both
nuclear and high-energy physics.
%(For details see NSAC Position Paper "Nuclear Theory and the U.S. Experimental Neutrino Physics Program", Alvarez-Ruso, L., Mahn, K., Mariani, C., Morf\'{i}n, J.G., Palamara, O., Zeller, G,).  
%Current event generators employ models of the nucleus that are far too simple in that they neglect, among other things, nucleon-nucleon interactions and do not agree with experiment~\cite{AguilarArevalo:2010zc,Boyd-Dytman}.  

Nuclear theorists are providing far more realistic nuclear models than
the relativistic Fermi gas, which has been the standard in event
generators, however, many aspects of the theory are still not at the
required level.  Event generators have begun the process of including
these improved although still incomplete advances, but manpower
problems are a hindrance. Thus, agreement of generator predictions
with data still remains a challenge and there is concern with the
accuracy of neutrino energy
estimations~\cite{Martini:2011wp,Lalakulich:2012hs,Shneor:2007tu}.
%Unfortunately even Better physics based approaches taking account of nucleon-nucleon interactions both agree with data and show that assignment of the incident neutrino energy based on simple models to be in error by as much as 200 MeV. 
This increased uncertainty in assigning the incident neutrino energy
will become a serious issue for oscillation measurements in the future
as statistics increase and the need for greater precision is
required~\cite{1412.4294}.  Furthermore, the HEP oscillation program
requires current neutrino nucleus theoretical developments to be
extended to higher-A such as Ar, to more relativistic energies and
momenta and to include the yield of resonance production cross
sections above the delta.  Significantly, this important work by
nuclear theorists has to be packaged in a form that can be swiftly
incorporated into neutrino event generators. Although it is not yet
clear how to do this, extensive collaboration between nuclear
theorists, builders of event generators and neutrino
experimentalists will be required. To achieve this, communication
between the NP and HEP communities should be improved so that new
theoretical advances can be more readily adopted.

In addition, the theory discussed above only describes inclusive cross
sections. Calculating the wide variety of exclusive final states
exceeds any current capability. Dealing with final state interactions
(FSI) will require extensive recourse to phenomenology.  Significantly
more complete higher energy neutrino cross section data will be
needed, especially for argon targets such as the proposed
CAPTAIN-MINERvA collaboration.  Additional input will likely come from
electron scattering carried out at kinematics similar to those
encountered in neutrino experiments. The electron beam at
Jefferson Laboratory is ideal and there is a recent JLAB
proposal~\cite{Benhar:2014nca} to carry out (e,e’p) measurements on
$^{40}Ar$ further illustrating the need for HEP/NP collaboration.  The
event generators must then incorporate these data into appropriate
models for existing and planned experiments.
%To most efficiently accomplish this ambitious program will require additional financial support of nuclear physics theorists working in this area.
  


%It should be obvious that the critical role of neutrino nucleus event generators needs to be emphasized and more community resources devoted to keeping them widely available, accurate, transparent, and current.
%In collaborations between neutrino experimentalists and nuclear theorists, these more advanced models need to be implemented quickly and consistently in the detector event generators employed in the analysis of experimental results. 
%as was done with the CCQE measurements on nuclear targets performed by the MiniBooNE experiment~\cite{AguilarArevalo:2010zc}. Only after taking into account two-nucleon mechanisms (two-particle-two-hole excitations) has it been possible to reconcile these experimental results with the, admittedly limited, information on the nucleon axial form-factor available from neutrino scattering on deuterium and pion electroproduction~\cite{Amaro:2010sd,Nieves:2011yp,Martini:2011wp}.  These findings have important implications for neutrino-oscillation experiments as a source of systematic error in the determination of the neutrino energy, which is not known for the non-monochromatic neutrino beams.  

%The development of neutrino detectors with high-resolution capability such as fine-grained sampling detectors and liquid argon has called special attention to dealing with so called final state interactions (FSI). While the theory discussed above is capable of calculating inclusive cross sections~\cite{Lovato:2014eva,1412.3081,Lovato:2015qka}, calculating the variety of exclusive final states that can be observed exceeds any current capability. Extensive recourse to phenomenology will be required and most of the necessary input will likely come from electron scattering carried out in kinematic regimes similar to those encountered in the neutrino beams. In that regard the electron beam at the Jefferson Laboratory is ideal and there exists a recent Jlab PAC approved proposal to carry out an (e,e’p) measurement on 40Ar that will be very informative. Furthermore, 

%{\sc The Present and Future Neutrino Interaction Experimental Program}
It is critical to benchmark generators against both
accelerator-based neutrino-nucleus interaction measurements and, via a
collaborative HEP and NP effort, electron-nucleus interaction
measurements.
%Recent years have witnessed intense experimental activity aimed at a better understanding of neutrino interactions with nucleons and nuclei. 
%A wealth of data exists through measurements of CC and NC processes made by ArgoNeuT~\cite{Acciarri:2014}, MINERvA~\cite{minerva}, MiniBooNE~\cite{miniboone}, MINOS near detector~\cite{minos}, NOMAD, SciBooNE~\cite{sciboone}, and T2K near detectors~\cite{Abe:2014nox}. In addition, the MicroBooNE experiment and NOvA near detector are beginning operation. 
The current experimental neutrino interaction program %(MINERvA, NOvA-ND, MicroBooNE, T2K Near Detector)
continues to provide important data and should be supported to its
conclusion.  To further refine the nuclear model in event generators,
future neutrino interaction measurements that span a range of neutrino
and antineutrino beam energies as well as target materials will be
needed to bring the current knowledge to the level required to reach
the goals of the long baseline oscillation program. The progress in
developing a broad international GENIE collaboration with a core group
at FNAL is encouraging.
%Such new experiments %have been proposed (CAPTAIN-MINERvA, LAr1-ND~\cite{Adams:2013uaa}, nuPRISM~\cite{Bhadra:2014oma}) and, 
%if approved, should be supported to conclusion. Current and future long-and-short- baseline neutrino oscillation programs should evaluate what additional neutrino nucleus interaction data is required to meet their ambitious goals and support experiments that provide this data. 

%Although this activity has been stimulated mostly by the needs of neutrino oscillation experiments in their quest for a precise determination of neutrino properties, the relevance of neutrino interactions with matter extends over a large variety of topics, including hadronic and nuclear physics.  Neutrino cross section measurements permit the investigation of the axial structure of the nucleon and baryon resonances, enlarging our views of hadron structure beyond what is presently known from experiments with hadronic and electromagnetic probes and lattice QCD. In the recent past, the electromagnetic form factors of the nucleon have been extensively studied at JLab with unexpected results such as the differing dependence of the electric and magnetic proton form factors over the $Q^2$ range from 1 to 8.5~GeV$^2$ and their deviation from the approximate dipole behavior at $Q^2 > 1$~GeV$^2$~\cite{Perdrisat:2006hj}. In contrast, the experimental information about the axial nucleon form factors is scarce. The existing data from neutrino quasi-elastic scattering on deuterium and pion electroproduction are compatible with a dipole behavior which is not well justified from a theoretical point of view~\cite{Bhattacharya:2011ah}. {\bf This points out the need for a feasibility study of a high-statistics deuterium experiment.  We are still dependent on low-statistics deuterium bubble chamber results from over 30 years ago.} 

%Understanding the excitation spectrum of the nucleon, and the properties of baryon resonances in general,  is a central question in strong interaction physics. Our knowledge about this spectrum was originally provided by elastic pion-nucleon scattering however recent experiments at MAMI, ELSA and JLab with photons and electrons have also unraveled the electromagnetic properties of baryon resonances. In contrast, the axial sector is practically unknown. Progress in this direction requires new and more precise measurements of neutrino inelastic scattering on hydrogen and deuterium targets.  {\bf This again stress the need for a new high-statistics experiment using a deuterium and hydrogen targets.}

%For nuclear physics, the fact that neutrino experiments are performed with nuclear targets represents a challenge but also an opportunity as neutrino-nucleus interactions incorporate new and important information due to the presence of both  axial and vector currents. They provide an excellent testing ground for models of the axial response. It is interesting to elucidate the role of multi-nucleon mechanisms, in particular meson exchange currents which are different than in the case of electromagnetic probes; the same is true for long-range correlations because the axial current gets renormalized in the nuclear medium in a different way than the vector current. Modern experiments, most notably MINERvA~\cite{Tice:2014pgu}, will measure different nuclear targets with the same (anti)neutrino flux, and the suite of current and proposed experiments include light (CH,H$_2$O) and heavy nuclei (Ar, Fe and Pb), which will provide important clues about the mass dependence of different observables. Novel approaches are also being investigated experimentally, such as the generation of pseudo-monochromatic beams on a single target material akin to electron-scattering (nuPRISM detector~\cite{Bhadra:2014oma}).

%{\sc Conclusion}
Understanding the subtleties of the nuclear environment and their
effect on neutrino experiments needs the input of nuclear physics
theorists specializing in this topic. It is important to create an
established procedure that allows nuclear theorists to join neutrino
generator experts and neutrino experimentalists in working toward this
goal. NuSTEC (Neutrino Scattering Theory Experiment Collaboration,
http://nustec2014.phys.vt.edu) has been established directly to
provide this environment.
%NuSTEC is encouraging the participation of individual nuclear theorists and neutrino experimentalists to explore neutrino-nucleus scattering with the goal of producing improved models that can be employed in neutrino-nucleus scattering event generators such as GENIE.  
A current example of such a project is the work on Green's Function
Monte Carlo techniques by a collaboration of Argonne, Jefferson Lab,
and Los Alamos nuclear theorists with Fermilab neutrino
experimentalists.  A white paper (Lovato, A. et. al. "Quantum Monte
Carlo Methods for Neutrino-nucleus Interactions",
FERMILAB-FN-0997-ND-T, JLAB-THY-15-2012, LA-UR-15-21054) has been
submitted to DOE-HEP and DOE-NP and is now being expanded to a
full proposal to be submitted to a possible HEP-NP FOA.

A summary of this Neutrino Interactions working group's discussions in
the form of a prioritized list follows:
\begin{enumerate}
\item The Neutrino-Nucleus Interaction is the least understood
  component of a detector’s response to neutrinos.

\item Improvements of nuclear models by nuclear theorists are
  essential. This can most efficiently be accomplished with additional
  financial support of NP theorists working in this area to provide a
  more robust model to meet the requirements of the oscillation
  program.  Rapidly incorporating these improvements in event
  generators is equally important and requires a collaborative effort
  of the HEP and NP communities.

\item The current experimental neutrino interaction program (MINERvA,
  NOvA-ND, MicroBooNE, T2K Near Detector) continues to provide
  important data and should be supported to its conclusion.  This
  includes efforts to improve the precision with which the neutrino
  flux is known.

\item The critical role of neutrino nucleus event generators needs to
  be emphasized and more community resources devoted to keeping them
  widely available, accurate, transparent, and current.
%It is critical to benchmark the generators against both accelerator-based neutrino-nucleus interaction measurements and, via a collaborative HEP and NP effort, electron-nucleus interaction measurements.  For example, expanded use of the existing Jefferson Laboratory data could bring significant insight.

\item Future neutrino interaction measurements such as the Fermilab
  short-baseline program (SBN) and the CAPTAIN-MINERvA experiment are
  needed to extend the current program of GeV-scale neutrino
  interactions.  The feasibility of a high-statistics deuterium
  experiment should be considered.  Current and future
  long-and-short-baseline neutrino oscillation programs should
  evaluate and articulate what additional neutrino nucleus interaction
  data is required to meet their ambitious goals and support
  experiments that provide this data.

\item Measurements and theoretical work are also needed to
  characterize neutrino interactions in the low energy regime ($ \leq
  100 $ MeV). This regime is especially relevant for core-collapse
  supernova neutrinos, and understanding is essential for development
  of future underground detectors.  This is also an area of
  collaboration where NP would bring in critical expertise.

\end{enumerate}
