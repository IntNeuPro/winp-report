\section{Neutrino Interactions}
\label{sec:Interactions}

In the United States, significant investment is being made using
accelerator-based sources of neutrinos to test the current 3-neutrino
mixing picture and determine whether or not neutrinos violate CP.  The
detectors associated with this program use heavy nuclei to achieve the
required event rates and carefully modeling the interaction of
neutrinos with these heavy nuclei is required.  Obviously then,
neutrino nucleus interactions necessarily play a central role in
accelerator based neutrino oscillation experiments and it has become
essential to create a more robust program of characterizing
neutrino-nucleus interactions and to integrate these results into
event generators.

Particularly now that the neutrino oscillation experiments are
evolving from the discovery to the precision stage, understanding the
challenging role of the nucleus in neutrino interactions has become
essential to help classify signal from background and minimize
systematic uncertainties in our neutrino oscillation investigations

 
Current event generators employ models of the nucleus that are far too
simple in that they neglect, among other things, nucleon-nucleon
interactions and do not agree with
experiment\cite{AguilarArevalo:2010zc,Boyd-Dytman}.  Nuclear
theorists can provide much better and well-tested nuclear models than
the relativistic Fermi gas used in most current event generators.
Better physics based approaches taking account of nucleon-nucleon
interactions both agree with data and show that assignment of the
incident neutrino energy based on simple models to be in error by as
much as 200~MeV\cite{Martini:2011wp,Lalakulich:2012hs,Shneor:2007tu}. This
increased uncertainty in assigning the incident neutrino energy will
become a serious issue for oscillation measurements in the future as
statistics increase and the need for greater precision is
required\cite{1412.4294}.  Furthermore, in order to be most useful to
the HEP oscillation program the current neutrino nucleus theoretical
developments must be extended to higher-A such as Ar, to more
relativistic energies/momenta and to include resonance production
cross sections.

In addition, the theory discussed above deals with calculation of
inclusive cross sections. Calculating the wide variety of exclusive
final states exceeds any current capability. Dealing with FSI will
require extensive recourse to phenomenology where most of the
necessary input will likely come from electron scattering carried out
at kinematics similar to those encountered in the neutrino
experiments. The electron beam at Jefferson Laboratory is ideal and
there is a recent JLAB proposal\cite{Benhar:2014nca} to carry out
(e,e’p) measurements on 40Ar further illustrating the need for HEP/NP
collaboration to advance this area.

Significantly, this important work by the nuclear theorists has to be
packaged in a form that can be swiftly incorporated in neutrino event
generators and it is not yet clear as to the best way to do this. It
will likely involve extensive collaboration between nuclear theorists,
the builders of event generators and neutrino experimentalists.

It is critical to benchmark the generators against both
accelerator-based neutrino-nucleus interaction measurements and, via a
collaborative HEP and NP effort, electron-nucleus interaction
measurements.  The current experimental neutrino interaction program
continues to provide important data and should be supported to its
conclusion.  To further refine the nuclear model in event generators,
future neutrino interaction measurements that span a range of neutrino
and antineutrino beam energies as well as target materials will be
needed to bring the current knowledge to the level required to reach
the goals of the long baseline oscillation program.

Understanding the subtleties of the nuclear environment and its
effects on what neutrino experimentalists measure in their detectors
can only be accurately performed with the input of nuclear physics
theorists specializing in this topic. It is important to have an
established procedure that promotes nuclear theorists to join neutrino
generator experts and neutrino experimentalists in working toward this
goal. NuSTEC (Neutrino Scattering Theorist Experimentalist
collaboration, http://nustec2014.phys.vt.edu) is a collaboration
established directly to provide this environment.  NuSTEC is
encouraging the participation of individual nuclear theorists and
neutrino experimentalists to explore neutrino-nucleus scattering with
the goal of producing improved models that can be employed in
neutrino-nucleus scattering event generators such as GENIE.  A current
example of such a project is the work on Green's Function Monte Carlo
techniques by a collaboration of Argonne, Jefferson Lab, and Los
Alamos nuclear theorists with Fermilab neutrino experimentalists.  A
white paper (Lovato, A. et al "Quantum Monte Carlo Methods for
Neutrino-nucleus Interactions", FERMILAB-FN-0997-ND-T,
JLAB-THY-15-2012, LA-UR-15-21054) has been submitted to the DOE-HEP
and DOE-NP and is now being expanded to a full proposal to be
submitted to a possible HEP-NP FOA.

A summary of this working groups discussions in the form of a
prioritized list follows:
\begin{enumerate}
\item The Neutrino-Nucleus Interaction is the least understood
  component of a detector’s response to neutrinos.

\item Improvements of nuclear models by nuclear theorists are
  essential. This can most efficiently be accomplished with additional
  financial support of NP theorists working in this area to provide a
  more robust model to meet the requirements of the oscillation
  program.  Rapidly incorporating these improvements in event
  generators is equally important and requires a collaborative effort
  of the HEP and NP communities.

\item The current experimental neutrino interaction program (MINERvA,
  NOvA-ND, MicroBooNE, T2K Near Detector) continues to provide
  important data and should be supported to its conclusion.  This
  includes efforts to improve the precision with which the neutrino
  flux is known.

\item The critical role of neutrino nucleus event generators needs to
  be emphasized and more community resources devoted to keeping them
  widely available, accurate, transparent, and current. It is critical
  to benchmark the generators against both accelerator-based
  neutrino-nucleus interaction measurements and, via a collaborative
  HEP and NP effort, electron-nucleus interaction measurements.  For
  example, expanded use of the existing Jefferson Laboratory data
  could bring significant insight.

\item Future neutrino interaction measurements are needed to extend
  the current program of GeV-scale neutrino interactions.  The
  feasibility of a high-statistics deuterium experiment should be
  considered.  Current and future long-and-short-baseline neutrino
  oscillation programs should evaluate and articulate what additional
  neutrino nucleus interaction data is required to meet their
  ambitious goals and support experiments that provide this data.

\item Measurements and theoretical work are needed also to
  characterize neutrino interactions in the low energy regime (<100
  MeV). This regime is especially relevant for core-collapse supernova
  neutrinos, and understanding is essential for development of future
  underground detectors.  This is also an area of collaboration where
  NP would bring in critical expertise.
\end{enumerate}

 
