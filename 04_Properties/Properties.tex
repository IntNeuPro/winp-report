\subsection{Neutrino Properties}
\label{sec:Properties}

Neutrinos are the most enigmatic particles in the Standard
Model. Identifying their properties may go beyond merely filling out
the few missing entries in the Particle Data Book. The absolute value
of the neutrino mass has cosmological implications. The magnetic
moment of the neutrino is a sensitive probe of TeV-scale physics
beyond the Standard Model. And identifying the quantum field nature of
the neutrino, i.e. whether it is a Dirac or Majorana fermion, may
determine if Lepton Number is a fundamental symmetry of Nature, and
shed light on the long-standing puzzle of the abundance of matter in
the visible Universe.

\subsubsection{Absolute neutrino Mass}

Absolute values of the neutrino masses are among the last unknown
parameters of the new ($\nu$) Standard Model. 
The combined results
of absolute neutrino mass searches,
neutrinoless double beta decay searches, and cosmology provide an
extraordinary constraint on the neutrino mass spectrum and models of the
neutrino mass.
Several experiments
have been discussed at the workshop: KATRIN, Project 8 and electron
capture in $^{163}$Ho. In the US, direct measurements of the neutrino
mass are generally supported by DOE-NP and NSF.   

KATRIN represents the state of the art of the presently available
technology. It will reach its ultimate sensitivity ($0.2$~eV at 90\%
C.L., $0.35$~eV for a 5$\sigma$ discovery) by the beginning of the
next decade. This sensitivity is comparable to the precision
available from the cosmological constraints and neutrinoless
double-beta decay, expected on a similar timescale. KATRIN is
currently in commissioning, and is expected to start operations with
the tritium source in 2016.

Techniques that may ultimately exceed KATRIN precision are in
development. Project-8 is a novel idea of measuring the momentum
spectrum near the tritium end-point using microwave cyclotron
radiation spectroscopy~\cite{Monreal:2009za}. The proof-of-principle
demonstration with a $^{83m}$Kr source has recently been
reported~\cite{Asner:2014cwa} in a small-volume trap. First operations
with a tritium source are planned and sensitivity similar to KATRIN is
possible by the end of the decade. Ultimately, a large
$\mathcal{O}(5~\mathrm{m}^3)$ volume experiment with an atomic tritium
source may start approaching the sensitivity to the neutrino mass
below the inverted hierarchy region. On the timescale of next decade,
such measurement would be complementary to direct measurements of the
hierarchy at ELBNF and cosmological constraints on the neutrino mass
of similar sensitivity.

Experiments looking to measure the end-point of the electron capture
spectrum, e.g. in $^{163}$Ho using bolometric micro-calorimeters are
in development (ECHO and HOLMES in Europe, NuMECS in the US). These
experiments employ large arrays of low-mass bolometers with a
$^{163}$Ho source embedded in the absorber. Novel readout techniques
are being explored to enable multiplexed readout of the a large number
of channels; these efforts can benefit from synergy with the
large-scale CMB arrays (supported by DOE-HEP through the CMB-S4
experiments). Large-scale production of $^{163}$Ho sources at reactor
facilities needs further  development and may benefit from
reinvigoration of the domestic isotope production program. 
Micro-calorimeter experiments aim to scale to 10k--100k
channels in about a decade, promising to reach sensitivity to the
neutrino mass below $0.1$~eV.


\subsubsection{Coherent elastic neutrino-nucleus scattering (CEvNS)
  and neutrino magnetic moment}

Coherent Elastic Neutrino-Nucleus Scattering (CEvNS) is a yet-unobserved
process that is cleanly predicted by the Standard Model.  The US
community is leading the effort towards the first observation of this
process, which may be possible with relatively modest investments in
the next few years. Coherent scattering experiments are also
sensitive to an anomalous magnetic moment of the neutrino. Detection
requires low energy thresholds and high-intensity neutrino sources,
thus there is complementarity with several other programs (neutrino
scattering, dark matter detection). Once observed, CEvNS could help
constrain models with non-standard neutrino interactions, have
sensitivity to nuclear weak charge, and would offer complementary
constraints on the sterile neutrino sector. 

A variety of source/detector configurations are being considered in the
search for CEvNS. Measurements with low-energy sources would rely
on the low-threshold detectors developed for dark matter searches at
reactors (e.g. RICOCHET) or with high-intensity radioactive
sources (e.g. $^{51}$Cr deployed in LZ). Measurements at higher
energies at accelerators (e.g. COHERENT at SNS and CENNS at Fermilab)
would use more conventional technologies. Searches for CEvNS match
well to the parameters of the program discussed at this workshop: they
can result in the first observation and sensitive limits on the
neutrino magnetic moment within 5--10 year timeframe, and with modest
investments ($<$\$10M). 

The RICOCHET experiment leverages significant R\&D and engineering on
low energy threshold detectors for SuperCDMS. It would deploy a tower
of six SuperCDMS detectors near a research reactor. With energy
thresholds as low as 100~eV$_{nr}$ already achieved at SuperCDMS,
several thousand events per month could be observed, depending on the
reactor and distance to the core. A critical issue is understanding
the in-situ backgrounds and could benefit from synergy with
short-baseline searches for sterile neutrinos at reactors.  Proponents
expect first deployment in the next few years and results on a 5--10
year timescale.  RICOCHET Phase-II~\cite{Formaggio:2011jt} would
operate underground with an intense electron capture source
(e.g. $^{51}$Cr), which requires development of even lower threshold
detectors and large active mass.

Another possibility for deployment of an intense electron capture
source such as $^{51}$Cr is the veto region of the LZ dark matter
detector~\cite{Coloma:2014hka}. With a 5~MCi source 1--2m from the
active volume and the low LZ energy threshold, detection
of CEvNS is possible with high significance. The experiment could
also place limits on the neutrino magnetic moment in the range
of (3--4)$\times10^{-12}\mu_B$, better than any terrestrial limit to
date and comparable in sensitivity to astrophysical constraints. This
effort is complementary with the sterile neutrino program and
could benefit from reinvigorated US isotope production and
isotope enrichment programs.


Experiments with intense accelerator-based sources, such as the
COHERENT~\cite{Akimov:2013yow} experiment with a DAR source at SNS and
CENNS~\cite{Brice:2013fwa} at the BNB facility at Fermilab offers
perhaps the most expeditious way of detecting CEvNS. COHERENT
would deploy multiple neutrino detectors in the vicinity of an intense
neutrino source from the SNS. Ge detectors (a test module from
the {\sc Majorana Demonstrator\/} neutrinoless double-beta decay
experiment), CsI crystals, as well as a 100~kg LXe 2-phase TPC are
being considered. The collaboration has identified several possible
deployment sites at SNS and has measured in-situ backgrounds.  The
Phase-I program is  ongoing, with the first neutrino scattering
results expected this year, and a possibility to discover CEvNS
within three years. The collaboration is receiving generous support
from ORNL and commitment from other national labs, as well as in-kind
contributions from international partners. 


\subsubsection{Majorana/Dirac nature of neutrinos (Neutrinoless
  Double-Beta Decay)}

Searches for Neutrinoless Double-Beta Decay
($0\nu\beta\beta$) aim to discover whether Lepton Number is a
fundamental symmetry of nature or is violated, and to determine the Dirac or
Majorana nature of neutrinos. The current generation of experiments
will search for $0\nu\beta\beta$  with a sensitivity to the effective Majorana
mass of order 100 meV. The next generation of experiments will aim for
an order of magnitude improvement in sensitivity to the effective
Majorana mass. With a "tonne-scale" isotopic mass and backgrounds at
the level of $<1$ per tonne per year in the region of interest, the
next-generation experiments  can  discover
$0\nu\beta\beta$ if it proceeds via light Majorana neutrino exchange
and if the lightest neutrino 
mass is above $\sim50$ meV, or if the spectrum of neutrino masses is
``inverted''. Even if neither of these two conditions is respected in
Nature, a discovery is possible if other mechanisms contribute to the
decay. While the potential for discovery of the Lepton Number
Violation in $0\nu\beta\beta$ is independent of the neutrino mass
hierarchy, sensitivity to the smallest possible masses consistent with
the inverted hierarchy is a milestone goal: with cosmological and
terrestrial constraints on the neutrino masses and the mass hierarchy
expected next decade, the next-generation  $0\nu\beta\beta$
experiments can be definitive.  

Planning for the next-generation tonne-scale  $0\nu\beta\beta$
experiment now is timely, and would help maintain US leadership in
the race for a Nobel-caliber discovery. The US community is
gearing towards selecting the 
leading candidates for one or more next-generation experiments,
shepherded by the NSAC-NLDBD committee. 
A targeted program of R\&D
activities towards mature concepts of next-generation experiments
is one of the priorities identified by this committee. 

The R\&D effort towards definition of the next-generation
$0\nu\beta\beta$ experiment is proceeding at a vigorous pace. There is
possible synergy with the R\&D activities in other areas of particle
physics, including neutrino science. For instance, there is a general
need for reliable designs of high-voltage distribution systems in
noble-gas liquid and gas TPCs, which could benefit from cooperation
between $0\nu\beta\beta$, dark matter and ELBNF
experiments. Development of large liquid scintillator and WbLS
detectors, and in particular techniques for isotope loading,
improvements in light yield and transparency are common to
KamLAND-Zen, SNO+, \textsc{Theia}, NuDot and others. Low-background techniques,
including low-mass, low-noise custom electronics which could be
deployed near sensitive detector volumes could benefit both
$0\nu\beta\beta$ and dark matter efforts. Those efforts also benefit
from ongoing cooperation in improving radio-assay capabilities and
sensitivities. Novel sensor technology development cuts across
disciplines, from neutrino science to large-scale CMB experiments, to
$0\nu\beta\beta$ and dark matter experiments. Finally, availability of
a large quantity of isotopically enrichment material will be critical
for most future $0\nu\beta\beta$ experiments. Development of 
domestic isotopic enrichment capabilities would be beneficial to the
broad US-based nuclear physics program.
