\subsection{Source, Cyclotron and Meson Decay at Rest Neutrinos}
\label{sec:SourceCyclotronDAR}

Isotope and meson decay-at-rest (DAR) processes can be used to provide
very high-intensity sources of neutrinos and antineutrinos with
energies spanning a few MeV to a few hundred MeV. This energy range
is ideal for a number of physics measurements including sterile
neutrino searches with baselines of 10--100~meters, searches for and
the possible discovery of coherent elastic neutrino-nucleus
scattering (CEvNS), and neutrino cross section measurements relevant
for astrophysical processes such as supernova explosions. There are
trade-offs in cost, schedule, and sensitivity so a diverse program of
possible technologies and setups can cover the wide range of possible
physics opportunities. High-activity radioactive sources can be
produced at reasonable cost and, when coupled with large,
low-background detectors, can have good sensitivity to electron
neutrino and antineutrino oscillations to sterile neutrinos. Very high
rates of radioactive isotopes that produce higher energy neutrinos can
be continuously produced by high-intensity cyclotrons and lead to
conclusive studies of sterile neutrino oscillations. Spallation
neutron sources such as the SNS and the JPARC-MLF facilities are
copious sources of neutrinos from meson DAR in the spallation
production dumps. Small, fairly low-cost detectors with good low
energy capabilities can be used at these facilities to search for
the elusive CEvNS process. These facilities could also host very
sensitive sterile neutrino oscillation experiments using large
detectors in the 50--1000~ton scale. The sections below outline the
features of experiments using these sources including
information on cost, timescale, and physics coverage.

\subsubsection{Small-scale Experiments}

\noindent{\bf Radioactive Source Experiments}

\noindent Radioactive source experiments could be a cost effective way
to investigate electron antineutrino disappearance in the region of
the reactor anomaly. The SOX program~\cite{bib:SOX} will use
high-intensity radioactive sources of neutrinos or antineutrinos to
investigate disappearance oscillations. The initial data run would
use sources placed below the Borexino detector. Details of the cerium
source production are described in~\cite{bib:CL}. This phase is
expected to have physics results within 5 years (cerium run will be
finished by the end of 2017, followed by a chromium run that should be
finished by mid 2018). The physics results are expected to probe at
95\% C.L. the entire Reactor Antineutrino Anomaly region. This phase
will also provide important R\&D for future upgrades relevant to SOX
and the US $^{51}$Cr program on LZ, SNO+, RICOCHET and possibly
others. This initial phase would have a cost in
the \$2--3M range and fit into the proposed FOA category. 
Future upgrades could include higher intensity sources and deployment
of the sources within the detector to get enhanced oscillation
parameter space coverage.

\noindent{\bf JPARC-MLF Pion/Kaon Decay-at-Rest Experiment}

\noindent JPARC-P56~\cite{JPARC_P_56} will directly probe the LSND
anomaly with $\bar\nu_e$ appearance using a 50~ton Gd-doped liquid
scintillator detectors at the JPARC-MLF (1~MW) 3~GeV spallation neutron
facility. First data is expected in the next 2--3 years. The
experiment will provide competitive (95\% C.L. coverage), but not
definitive, sensitivity to the LSND allowed region. The project is of
modest scale with Japanese costs at the \$5M level, but has the
potential for upgrades and additional detector modules. The main US
contribution to JPARC-P56 will be the dilute Gd-loaded liquid
scintillator, which will fit well into the FOA scope with an estimated
cost of \$1.5M.

The JPARC-MLF beam also uniquely allows the possibility to study kaon
decay-at-rest (KDAR)~\cite{KDAR} muon neutrinos 
and related physics for the first time with an expected sample of over 10$^5$ muon
neutrino charged current events. The KDAR muon neutrinos are
mono-energetic with an energy of 236~MeV. As the only relevant
known-energy muon neutrino above the charged current threshold, this
unique neutrino can be used for studying short baseline oscillations
indicative of a sterile flavor, neutrino cross sections relevant for
future CP violation searches, and nuclear physics with a known-energy,
weak-interaction-only probe.

\noindent{\bf Coherent Elastic Neutrino-Nucleus Scattering Experiments}

\noindent Coherent Elastic Neutrino-Nucleus Scattering (CEvNS) is an
unambiguous prediction of the Standard Model. Recent advances in
detector design now put this so-far elusive prize within reach. Such a
measurement will open the door to a host of new ways to better
understand neutrino properties, and to search for new physics. CEvNS
represents an eventually dominant background for dark matter
detection, and its measurement will demonstrate dark matter detector
response. As a neutral current process, it will be a new tool for
sterile neutrino oscillation experiments.

The COHERENT experiment~\cite{Akimov:2013yow} will search for CEvNS at the
SNS with three detector targets (CsI, Xe, Ge). The 1.4~MW SNS has a
flux of 3.3$\times$10$^{7}$ $\nu$ cm$^{-2}$ s$^{-1}$ at 20~m with a
clean pion DAR spectrum. A neutron measurement campaign has identified
several suitable deployment sites within 30 m of the source. Phase 1
of the experiment is likely to produce initial results within the
year, and would have a cost of \$2--3M, fitting well into the proposed
FOA category.

The CENNS experiment~\cite{Brice:2013fwa} will search for CEvNS at the
BNB with a redeployment of the MINICLEAN detector to Fermilab. The
32~kW BNB has a flux of 5$\times$10$^{5}$ $\nu$ cm$^{-2}$ s$^{-1}$ at
20~m. Measurements indicate that backgrounds from neutrons are
manageable with sufficient shielding, in a green field site. The first
results are expected after 2018, and are coupled to the physics
program of MINICLEAN. The expected cost is \$2M, which fits well into
the proposed FOA category.

\subsubsection{Mid-scale Experiments}

\noindent{\bf The IsoDAR Isotope Decay-at-Rest Experiment}

\noindent The IsoDAR experiment~\cite{IsoDAR,IsoDARJuno} will make a
highly definitive investigation of electron antineutrino disappearance
in the reactor anomaly region and make precision electroweak
measurements searching for neutrino non-standard interactions. The
experiment uses a very high intensity $^8$Li antineutrino source placed
near a large scintillator detector such as KamLAND or JUNO. IsoDAR
can also be coupled with WATCHMAN and provide an important component
of the WATCHMAN physics program. The cost of IsoDAR is estimated to
be \$30M so it is a mid-scale project. At this point, the development
of the IsoDAR cyclotron and neutrino source needs engineering R\&D
support at about the \$1M level to complete prototypes and prepare a
Conceptual Design Report to be submitted to the agencies.

\noindent{\bf The OscSNS Pion Decay-at-Rest Experiment}

\noindent OscSNS~\cite{OscSNS} has the capability to make a definitive
search for electron antineutrino appearance using a pion DAR beam with
much lower uncertainties than LSND. The cost is at the \$12M scale for
civil construction and \$8M for a new 800~ton liquid scintillator
detector with a start date 3 years after initiation. With the high
neutrino rate at SNS, the experiment can cover the full LSND signal
region in 2 years with some capability to see oscillatory behavior for
$\Delta$m$^2 > 1$~eV$^2$. The successful support by DOE NP for
infrastructure development at the SNS related to the fundamental
physics neutron beamline and support facility can serve as a model for
future infrastructure development related to OscSNS and other neutrino
efforts supported by DOE HEP.

\subsubsection{Cross Section Measurements using DAR Neutrino Sources}

\noindent DAR neutrino sources can be used to measure a number of
important neutrino interaction cross sections. Neutrino-nucleus
cross-section measurements on various targets are inputs to supernova
modeling and for understanding supernova detectors. Specifically, the
CENNS, CAPTAIN-BNB~\cite{Berns:2013usa}, JPARC-MLF, and COHERENT (Ge,
CsI, Xe) at SNS experiments will address many of these cross-section
measurements. For example, ongoing measurements of the
neutrino-induced neutron background for COHERENT play an important
role in r-process nucleosynthesis, as well as in the HALO SNe
experiment~\cite{halo}. To make these measurements one needs to
understand the flux spectrum, the detector characteristics and the
backgrounds. Neutron backgrounds are the most important and need to be
addressed by location or shielding.
