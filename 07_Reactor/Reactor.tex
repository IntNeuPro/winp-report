\section{Reactor Neutrinos}
\label{sec:Reactor}
Reactor neutrinos have played a central role in the history and our understanding of neutrinos.  From the first experimental observation of the neutrino by Reines and Cowan at the Savannah reactor in the 1950's to the demonstration of neutrino oscillations with KamLAND and the recent precision measurement of the neutrino mixing angle $\theta_{13}$, reactor neutrinos have provided us with a unique tool for discovery for over five decades. Reactor experiments have advanced our understanding of the 3-neutrino framework and enabled the most precise measurement of neutrino mixing.

Emission of electron antineutrinos from reactors provides a flavor-pure source of antineutrinos with energies up to $\sim$ 8 MeV. Reactor neutrinos are studied at distances as close as several meters from the reactor core and detected up to hundreds of kilometers from a nuclear power plant.  The abundant flux of antineutrinos from a nuclear reactor allows experiments to probe for new physics such as sterile neutrinos and neutrino magnetic moments, provide an opportunity to observe coherent neutrino scattering,  and make a determination of the neutrino mass hierarchy. The detection and study of reactor antineutrinos allows the monitoring and study of the power and fuel composition of reactors and finds application in non-proliferation and safeguards. 

Over the next decade neutrino experiments will make precision tests of 3-neutrino oscillations, aim to understand anomalous signatures in the current suite of neutrino data, determine the mass hierarchy and search for new CP violation. Reactor experiments will play a unique and complementary role in this program and provide discovery potential of new physics at very modest costs. The observation of sterile neutrinos or other new physics would be a paradigm shift and set the course of particle physics for years to come. The resolution of the neutrino mass hierarchy will provide crucial input to determing
the nature of neutrino which.  Precision measurements of neutrino mixing parameters and knowledge of the mass hierarchy will help maximize the physics reach of the long-baseline experiment planned in the US. 

Measurements of the reactor neutrino flux and spectrum compared to recent models of reactor antineutrino production have revealed discrepancies in both the total measured reactor antineutrino flux as well as the energy spectrum of neutrinos. The observed flux is found to be $\sim$5-6\% low while the recent $\theta_{13}$ experiments have revealed a distortion in the 4-6 MeV region of the spectrum.  The observed discrepancies may point to new physics such as eV-scale sterile neutrinos
or highlight incomplete nuclear models in the predictions of the antineutrino flux from reactors. Only a well-controlled measurement of the antineutrino flux and spectrum at short-baselines can directly  address these questions. A search for antineutrino oscillations over meter-long baselines probes  the hypothesis of sterile neutrinos in the appropriate mass region while a measurement of the flux and  spectrum at a research reactor with highly enriched $^{235}$U fuel tests our understanding of antineutrino emission in nuclear fission. Reactor neutrinos also provide a clean tool for the study of neutrino oscillations over baselines of several meters to hundreds of kilometers. Over distances of some fifty kilometers the observed oscillated energy spectrum is sensitive to the effect of the neutrino mass hierarchy independent of the matter effect. This provides a unique method for the determination of the mass hierarchy along with precision measurements of neutrino mixing parameters.

In light of these consideration the reactor working group has identified the following near-term priorities:

{\em A short-baseline experiment designed to resolve the reactor neutrino anomaly through oscillation 
and spectral measurements has the potential to discover new physics in a very cost effective manner and is the highest priority of this working group.}  Short-baseline reactor disappearance experiments are complementary to the FNAL short-baseline program focusing on appearance measurements. Given the US experience and available facilities, there is an excellent opportunity for the US to host and lead a short-baseline reactor experiment. The proposed projects are ready to proceed and provide an opportunity for world-leading, high-impact science at modest cost of several M\$ in the next three to five years. The experiments offer opportunities for international collaboration with Canada and China and possibly Europe. The timely execution is critical to guarantee the highest impact in the international context and maximize the discovery potential. 

{\em Medium-baseline experiments plan to determine the neutrino mass hierarchy without the matter effect and precisely measure $\theta_{12}$, $\Delta m^2_{21}$, and $\Delta m^2_{32}$. Near-term R\&D can inform a potential US contribution overseas and ensures US  leadershipin the determination of the mass hierarchy. } A reactor experiment at medium baselines represents an excellent opportunity to continue the long-standing US-China collaboration.

Measurements of reactor neutrinos are also relevant to nuclear physics and applied reactor safeguards and the proposed experiments will continue the long-standing US expertise in this area.  Many of the theoretical and experimental challenges are common across these fields and reactor neutrino measurements have the potential to uniquely inform these communities.
