\pagebreak
\subsection{Reactor Neutrinos}
\label{sec:Reactor}

Reactor neutrinos have played a central role in the history and our
understanding of neutrinos.  From the first experimental observation
of the neutrino by Reines and Cowan at the Savannah reactor in the
1950's~\cite{Reines:1960pr} to the demonstration of neutrino
oscillations with KamLAND~\cite{Abe:2008aa} and the recent precision
measurement of the neutrino mixing angle $\theta_{13}$
~\cite{Abe:2012tg, An:2013zwz}, reactor neutrinos have provided us
with a unique tool for discovery for over five decades. Reactor
experiments have advanced our understanding of the 3-neutrino
framework and enabled the most precise measurement of neutrino mixing.

Emission of electron antineutrinos from reactors provides a
flavor-pure source of antineutrinos with energies up to
$\sim$8~MeV. Reactor neutrinos are studied at distances as close as
several meters from the reactor core and up to hundreds of kilometers.
The abundant flux of antineutrinos from a nuclear reactor allows
experiments to probe for new physics such as sterile neutrinos and
neutrino magnetic moments, observe coherent neutrino scattering, and
determine the neutrino mass hierarchy. The detection and study of
reactor antineutrinos allows the monitoring and study of the power and
fuel composition of reactors and finds application in
non-proliferation and safeguards.

\subsubsection{Physics Reach and Discovery Potential}
Over the next decade neutrino experiments will make precision tests of
3-neutrino oscillations, aim to understand anomalous signatures in the
current suite of neutrino data, determine the mass hierarchy and
search for new CP violation. Reactor experiments will play a unique
role in this program and provide discovery potential of new physics at
modest costs. The observation of sterile neutrinos or other new
physics would be a paradigm shift and set the course of particle
physics for years to come.  The resolution of the neutrino mass
hierarchy is fundamental to understanding the neutrino mass spectrum
and will provide important input to the search for neutrinoless double
decay. Precision measurements of neutrino mixing parameters and
knowledge of the mass hierarchy will help maximize the physics reach
of long-baseline experiments and are fundamental to any extensions to
the Standard Model.

Measurements of the reactor neutrino flux and spectrum compared to
recent models of reactor antineutrino production have revealed
discrepancies in both the total measured reactor antineutrino flux as
well as the energy spectrum of neutrinos. The observed flux is found
to be $\sim$5-6\% low~\cite{Mention:2011rk, Zhang:2013ela} while the
recent $\theta_{13}$ experiments have revealed a distortion in the 4-6
MeV region of the spectrum.  The observed discrepancies may point to
new physics such as eV-scale sterile neutrinos and already indicate
that nuclear models in the predictions of the antineutrino flux from
reactors are incomplete.  Only a well-controlled measurement of the
antineutrino flux and spectrum at short-baselines can directly address
these questions. A search for antineutrino oscillations over
meter-long baselines probes the hypothesis of sterile neutrinos in the
appropriate mass region while a measurement of the flux and spectrum
can inform our understanding of antineutrino emission from reactors.
Over distances of some fifty kilometers the observed oscillated energy
spectrum provides a unique method for the determination of the mass
hierarchy along with precision measurements of neutrino mixing
parameters.

\subsubsection{Reactors and Experimental Facilities}
The US has several reactor facilities that are well-suited for
short-baseline experiments at distances of $\mathcal{O}$(10~m). The
NIST, HFIR, and ATR reactors in the US operate at 20-250~MW with
highly enriched $^{235}$U, provide easy access and technical support
for a world-leading short-baseline reactor neutrino experiment. Other
compact, high-powered sources for antineutrinos such as a naval
reactor are being explored. Through international collaboration the US
neutrino physics community also has an exciting opportunity to
participate in a medium baseline experiment overseas near the Taishang
and Yangjiang reactors in China.

% The reactor working group has identified the following opportunities for an intermediate neutrino program:

\subsubsection{Small-Scale \& Short-Baseline: Sterile Neutrinos and Reactor Spectrum}
{\em A domestic short-baseline experiment designed to resolve the
reactor neutrino anomaly through oscillation and spectral measurements
has the potential to discover new physics in the next 3--5 years at
modest costs of 3--4\$M.}  Proposed short-baseline reactor
disappearance experiments such as NuLAT~\cite{Lane:2015alq} and
PROSPECT~\cite{Ashenfelter:2013oaa} are complementary to the FNAL
short-baseline program focusing on appearance measurements, and will
answer both the question of short-baseline oscillation and our
understanding of the reactor antineutrino spectrum. The proposed
projects have deployed test detectors at NIST and HFIR, are ready to
proceed with full design and construction, and provide an opportunity
for world-leading, high-impact science at modest costs. In a
technically limited schedule, data taking can begin in 2016 with first
physics results in 2017. Given US experience and available reactor
facilities, the US is in an excellent position to host and lead a
short-baseline reactor experiment. The experiments offer opportunities
for international collaboration with Canada, China, and Europe.
Several other efforts reactor experiments worldwide are at various
stages of development. This includes SOLID, STEREO, DANSS, and
Neutrino-4. Experiments proposed in the US aim to optimize physics
sensitivity to both the reactor spectrum and neutrino oscillations,
proceed in a phased approach, and provide comprehensive systematic
control through novel scintillator and detector technologies, movable
detectors, and extensive background control. Detectors proposed in the
US offer the highest energy resolution of $\sim$4.5\% which provides
unmatched capability in the measurement of the reactor spectrum. This
comprehensive and phased approach will maximize the discovery
potential, limit technical risk, and provide flexibility to respond to
future discoveries.  Timely execution is critical to guarantee the
highest impact and facilitate European participation in a US
experiment.

\subsubsection{Mid-Scale \& Medium-Baseline: Mass Hierarchy and Oscillation Parameters}
 {\em Medium-baseline experiments aim to determine the neutrino mass
hierarchy without matter effect and precisely measure $\theta_{12}$,
$\Delta m^2_{21}$, and $\Delta m^2_{32}$. over the next 7--10
years~\cite{Kettell:2013eos}. A US contribution can make a critical
impact to one of these overseas experiments.} The JUNO experiment in
China and RENO-50 in Korea are designed to determine the neutrino mass
hierarchy and precisely measure oscillation parameters. In particular,
the JUNO experiment, hosted by China and with substantial European
contributions, offers strong leveraging of the successful US-China
collaboration. The US is well-positioned to make an important
contribution to JUNO.  Near-term R\&D of a few \$M can help define US
scope focused on the calibration system with an eventual US JUNO
contribution of order \$20M over 5--8 years.  A reactor experiment at
medium baselines represents an excellent opportunity to continue the
long-standing US-China collaboration and ensures US leadership in the
determination of the mass hierarchy.

\subsubsection{Synergies} 
{\em Measurements of reactor neutrinos are also relevant to nuclear
physics and applied reactor safeguards. The proposed experiments will
continue the long-standing US expertise in this area.} Short-baseline
experiments provide opportunities for detector R\&D into technologies
such as novel scintillators and highly segmented detectors. Large
reactor antineutrino detectors offer synergistic opportunities for
geo-neutrino physics and astrophysics. Many of the theoretical and
experimental challenges are common across these fields and reactor
neutrino measurements have the potential to uniquely inform these
communities.
