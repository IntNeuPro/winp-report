\section{Neutrino Detector R\&D}
\label{sec:RandD}

\subsection{Water and Liquid Scintillator}

The development of new scintillator materials and doping agents has proven critical to the advancement of neutrino detector design.  Further development of these materials is a critical step for future experiments; supporting this effort should be a high priority in the intermediate program.  This program includes target development and also characterization, including: light yield and timing measurements at low and high energy, and attenuation measurements.

The newly-developed water based liquid scintillator (WbLS) could enable a massive detector with a broad physics program at relatively low cost.  A particularly nice feature is the potential to separate fast, directional Cherenkov light from the slower yet far more abundant isotropic scintillation light.  Should this be achieved, this would enable astonishing advances in signal identification and background rejection capabilities via particle identification, resulting in vast improvements in physics reach.  This potential capability should be explored via both optimization of the WbLS target -- by modifying the LS fraction and thus relative magnitudes of the Cherenkov and scintillation components, by use of various additives to delay the scintillation light, or wavelength shifters to minimize absorption/reemission of Cherenkov light --  and alternate photon detection methods.  The ability to reconstruct event energy and direction in (Wb)LS needs to be demonstrated both theoretically (in simulation) and in practice (in smaller scale experiments). 

Large water Cherenkov and (water-based) scintillator detectors require very high purity target liquids.  Purification techniques for water are well understood in industry and need no R\&D, but further development is required for (Wb)LS purification.  At the same time, a program to determine compatibility of construction materials with (Wb)LS must exist for future detectors.

Isotope loading in traditional liquid scintillator, Gadolinium doping for water detectors, and the potential to load metallic isotopes in WbLS broaden the potential physics program and enhances the sensitivity of these experiments significantly.  Techniques to load isotope while maintaining the optical purity of the target should continue to be developed.  %The above studies (light yield, timing, purification, materials compatibility) must also be performed for the isotope-loaded scintillator.

A driving cost and critical performance factor in large-scale water or scintillator detectors is the photomultiplier tubes. R\&D to produce low cost, large area, ultra-fast photon detectors is important for the neutrino community. Correspondingly fast, high precision readout will be critical to take advantage of developments in photon detector technology.  

Water-based detectors (including WbLS) have the advantage of a low cost detector medium allowing very large-scale experiments. Future experiments will be limited by the cost and excavation techniques for the cavern needed to house the experiments. R\&D to find lower-cost construction methods, including PMT deployment and readout techniques, can facilitate next-generation neutrino detectors. 

Several projects are underway that can address these topics.  These range from bench-top scale development and characterization, primarily at BNL but also at U. Chicago, U. Penn, LBNL, Iowa State, and MIT, to full-scale projects such as EGADS (Gd loading), ANNIE (fast timing), WATCHMAN phase II (WbLS deployment, fast timing), SNO+ (Te loading), and CHIPS (large-scale construction).  Ultimately such projects will inform the design of a massive future detector such as the proposed \textsc{Theia} experiment.


%A full program of development would incorporate:
%\begin{itemize}
%\item WbLS cocktail development
%\item WbLS cocktail characterization including: light yield, timing at high and low energy
%\item Attenuation measurements
%\item Isotope loading techniques
%\item Purification and recirculation methods
%\item Materials compatibility
%\item Ultra-fast, high precision photon detection methods (and associated readout electronics)
%\item Reconstruction techniques in WbLS (energy and direction) 
%\item Large-scale construction techniques, including: PMT deployment, readout, cabling
%\end{itemize}

\subsection{Liquid Argon}

Several of the ongoing and proposed experimental efforts will provide R\&D which either will substantially improve the understanding of the LAr TPC performance or potentially expand the capabilities of the  ELBNF experiment. The test beam measurements are especially important as they provide critical data for improving the detector model and understanding the detector's systematic errors. The majority of these experiments will either use the test beam facilities at FNAL or CERN or the neutrino beams at FNAL. The support for these efforts can be a combination of R\&D funding at FNAL, project funding for the SBN or ELBNF projects, funding from this mid-range program, or other laboratory funding. Careful evaluation and prioritization of these experiments by the FNAL PAC including an evaluation of impact of R\&D by these experiments is expected, and should provide important guidance to the selection process. However, the substantially larger funding available to the FNAL projects should also be taken into account. Only one experiment that will provide critical R\&D needed for the ELBNF program is outside the FNAL program: the neutron cross section measurements proposed by the CAPTAIN experiment are necessary to understand the detector response and energy resolution. 

The following factors should be taken into account when evaluating the impact different experiments could have on the long range program.

\begin{itemize}

\item  A comprehensive test beam program must be performed to characterize present and future LAr TPCs. This is necessary to calibrate the detector response of existing and future LAr detectors and to verify the systematic error estimates. This program should include electromagnetic and hadronic showers measurements, neutron cross section measurements, and energy deposition measurements with different charged particle beams typical for particles in future and present experiments. Experiments which could contribute to this are LArIAT, CAPTAIN, and the CERN neutrino platform experiments.

\item  R\&D on the generation and breakdown of high voltage will reduce the risk to future LAr detectors and could lead to more monolithic and lower cost detector designs based on longer drifts. The causes of HV breakdown in LAr are not well understood. If the process for HV discharge in LAr is better understood then detectors could be designed for higher voltages (if the electron lifetime is sufficiently large). This could lead to larger cheaper detectors and could enable dual-phase style detectors.  In addition the manufacture of LAr feedthrus for voltages above 100 kV has only been achieved successfully by a small number of groups.    

\item The process of contamination generation and transport inside the large liquid argon detectors is not well understood. Better modeling of the sources and migration of contaminates in large cryogenics systems will aid in designing future detectors.

\item  R\&D which improves the understanding of the generation and propagation of both light and charge in large LAr TPCs will improve the detector model for ELBNF.
 
\item Present photon detector designs capture a very small fraction of the scintillation light generated in the LAr detectors. Detectors with better light collection efficiency should be developed. 

\item The development of cold electronics for LAr detectors is critical. Advanced designs for cold Pre-Amps and ADCs exist but a control chip is now in early stages of development. Development of electronics to read out large arrays of SiPMs is necessary.

\end{itemize}




