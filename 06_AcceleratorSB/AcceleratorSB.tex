\section{Short Baseline Accelerator Neutrinos}
\label{sec:AcceleratorSB}

Accelerator decay-in-flight (DIF) neutrino beams provide an excellent
opportunity for near-term, cost-effective neutrino experiments
pursuing a variety of exciting physics and detector development goals.
The US is home to two existing neutrino beams, both located at
Fermilab.  The Booster neutrino beam (BNB) has the significant
advantage of being at shallow depth and parallel to the ground,
enabling the deployment of multiple detectors at different baselines
for relatively modest construction costs.  Improvements to the
performance of the BNB are currently under consideration and would
significantly strengthen all experiments that utilize this beam.
Preliminary studies indicate that a significant flux increase is
feasible by fitting a second horn in the existing beam facility,
although a detailed schedule and cost estimate needs to be prepared as
recently requested by the Fermilab Physics Advisory Committee
(PAC). New experiments on the Main Injector neutrino beam (NuMI) are
also possible, but are limited by existing space in the underground
near detector cavern.

A major element of the intermediate neutrino program will be the
Short-Baseline Neutrino (SBN) program of three liquid argon detectors
along the BNB~\cite{SBNProposal}.  The SBN program can resolve a class
of experimental anomalies in neutrino physics and perform the most
sensitive search to date for sterile neutrinos at the eV mass-scale
through both appearance and disappearance oscillation
channels. Additional physics includes the study of neutrino-argon
cross sections with millions of interactions using the neutrino fluxes
of the BNB (on-axis) and NuMI (off-axis) beams.  This program brings
together an international team of scientists and engineers to advance
the LAr-TPC technology for neutrino physics while utilizing its
capabilities to explore one of the exciting open questions in neutrino
physics today.  The SBN program was awarded Stage-1 approval at
Fermilab in February 2015.  Preparation of the three detectors is
proceeding rapidly in order to match a tight time schedule that
anticipates first data in 2018.  MicroBooNE is now installed on the
BNB and ready for commissioning, ICARUS is being refurbished at CERN
in preparation for operations at shallow depth, and detailed designs
are being developed for the near detector, LAr1-ND.

The Accelerator Neutrino Neutron Interaction Experiment
(ANNIE)~\cite{Anghel:2014ynd} is another proposal to utilize the
on-axis flux of the Booster neutrino beam.  ANNIE is a gadolinium
doped water detector instrumented with advanced photodetectors that
will measure neutron production in GeV-scale neutrino interactions,
something presently not well understood and that represents a limiting
factor in proton decay and supernova neutrino measurements in large
water detectors.  ANNIE also presents an opportunity to demonstrate
new Large Area Picosecond Photodetector (LAPPD)
technology~\cite{lappd}, now being commercialized through the DOE STTR
program, in the context of a neutrino detector. ANNIE was awarded
Stage-1 approval following the January meeting of the Fermilab PAC.

%ANNIE also presents an opportunity to demonstrate a novel new light collection technology in a neutrino detector, the Large Area Picosecond Photon Detectors (LAPPDs), now being developed.  ANNIE was also awarded Stage-1 approval following the January meeting of the Fermilab PAC.              

The CAPTAIN detector~\cite{Berns:2013usa}, a 5 ton LAr-TPC being built
at Los Alamos National Laboratory, anticipates running in both the BNB
and NuMI neutrino beams.  By positioning the detector far off-axis at
the BNB, CAPTAIN can study low energy neutrino interactions (10s of
MeV) to better understand neutrino interactions in the energy range of
supernova neutrinos that may be observed in future large LAr
detectors.  The CAPTAIN-MINERvA experiment involves positioning the
CAPTAIN detector in front of the existing MINERvA detector in the NuMI
near detector hall to make precision measurements of neutrino-argon
cross sections in the multi-GeV energy range.

%The nuPRISM detector represents a novel approach to substantially reducing the impact of neutrino cross section model uncertainties in long-baseline oscillation experiments and has been proposed as part of the near detector complex at J-PARC in Japan.  By sampling the neutrino fluxes at angles from 0-4 degrees off-axis, representing different ranges of neutrino energy, nuPRISM can untangle the complicated energy-dependent effects of neutrino-nucleus interactions, including scattering off multiple correlated nucleon initial states.  nuPRISM also has sensitivity to sterile neutrino oscillations by sampling the beam at the same $L$ but different $E_{\nu}$ ranges.    

The NuPRISM detector~\cite{Bhadra:2014oma} represents a novel approach
to substantially reducing the impact of neutrino cross section model
uncertainties in long-baseline oscillation experiments, and has been
proposed as part of the near detector complex at J-PARC in Japan. By
sampling the neutrino beam at angles from 1-4 degrees off-axis,
representing different ranges of neutrino energy, NuPRISM can map out
the relationship between neutrino energy and the observed lepton
kinematics. NuPRISM, in combination with the existing T2K near
detector, also has sensitivity to sterile neutrino oscillations by
sampling the beam at the same $L$ but different $E_{\nu}$
ranges. Finally, NuPRISM can construct mono-energetic beams to make
unique neutrino cross section measurements, such as the first ever
neutral current cross sections as a function of neutrino energy, and
the separation of traditional CCQE events from interactions involving
several nucleons.

Accelerator neutrino beam facilities and precision neutrino detectors
present another exciting opportunity for the interim physics program.
By running in a 'beam-dump' mode (steering the proton beam off the
target and into the beamline absorber), searches can be performed for
dark sector particle production with reduced neutrino backgrounds.
The MiniBooNE experiment has recently collected data in this mode and
is currently analyzing the
results~\cite{Dharmapalan:2012xp,Thornton:2014ufa}.  The LAr detectors
of the SBN program present an opportunity to perform similar searches
in the future with enhanced sensitivity.

The need to understand the physics being pursued with these
short-baseline accelerator beam experiments (anomalies, sterile
neutrinos, neutron production, neutrino-nucleus cross sections) and
the importance of developing detector technologies for the future
neutrino program motivate an aggressive time scale for each of these
experiments, making them ideal for an interim neutrino program on the
road to the next generation US long-baseline experiment.
