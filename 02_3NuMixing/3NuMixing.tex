\section{Three Neutrino Mixing}
\label{sec:3NuMixing}

Three neutrino mixing has been well established by a variety of
experiments yielding consistent results. Besides the overall neutrino
mass scale, there are three parameters that are still unknown in this
framework and which can be better addressed in neutrino oscillation
experiments: the neutrino mass ordering, $\delta_{CP}$, and the octant
of $\theta_{23}$.  The determination of these unknowns is of key
importance to understanding the mechanism responsible for neutrino
mass and potentially the generation of the matter-antimatter
asymmetry in the early universe.  The suite of presently running
experiments are either directly exploring the three unknowns, or are
producing results that will contribute to these measurements
by reducing systematic and theoretical uncertainties (neutrino
interaction measurements for example), or by precisely measuring
related oscillation parameters ($\theta_{13}$, $\theta_{23}$, $|\Delta
m^{2}_{\textnormal{atm}}|$).  Proposed and running experiments can
potentially measure the ordering and $\theta_{23}$ octant in the next
decade.  The measurement of $\delta_{CP}$ is likely to take longer and
require a dedicated large-scale experiment.  If discrepancies are
found in the measurement of one or more of these fundamental
parameters across different experiments, it could point to new
physics.

Recent results from the Daya Bay Reactor Neutrino
Experiment~\cite{Zhang:2015fya} include the most precise measurement
of $\sin^{2}2\theta_{13}$ to date and a measurement of $|\Delta
m^{2}_{\textnormal{atm}}|$ that is comparable in precision and
consistent with the long-baseline measurements.  By Daya Bay's
expected end date in 2017, it will provide the most precise
measurement of $\sin^{2}2\theta_{13}$ for many years to come.  The
Double Chooz experiment has recently installed their near detector,
which will greatly increase the precision of their
$\sin^22\theta_{13}$ measurements~\cite{Abe:2014bwa}. Double
Chooz has the unique ability to take reactor-off data due to having
only two reactor cores, allowing a strong constraint on the background
in their $\theta_{13}$ analysis.

The combined data from the MINOS long-baseline experiment using the
low-energy NuMI beam and atmospheric neutrino data in the MINOS far
detector provides the most precise measurement of the atmospheric mass
splitting~\cite{Adamson:2014vgd}.  MINOS+ is currently running in the
medium-energy NuMI beam.  Including these data in the fit will further
improve the precision.  Furthermore, the high statistics in the energy
region just above the oscillation maximum puts MINOS+ in a unique
position to test the validity of the three neutrino mixing model. T2K
can also measure the atmospheric parameters and has made the most
precise measurement of $\sin^2\theta_{23}$.  T2K also made the first
observation of electron neutrino appearance.  Assuming the value of
$\theta_{13}$ from reactor data, T2K can exclude some values of
$\delta_{CP}$ at the 90\% confidence level~\cite{Abe:2015awa}.  The
NOvA experiment at Fermilab has only just started running, but has the
potential to determine the octant or the mass ordering within some
range of the parameters in particular in combination with data from
both T2K and MINOS+.

Super-K continues to use atmospheric neutrino data to address
the three main questions in three neutrino mixing and has recently
included combined fits with T2K data~\cite{Wendell:2014dka}.  A
proposed upgrade for Super-K in the intermediate program would involve
adding gadolinium to the water to identify electron antineutrinos via
the inverse beta decay interaction~\cite{Beacom:2003nk} enabling
detection of diffuse supernova neutrinos. IceCube has recently
presented results on the atmospheric mixing parameters that are
consistent with and comparable in precision with results from
Super-K~\cite{Aartsen:2014yll}.

It is important to realize that precision measurements of oscillation
parameters made by currently running experiments can have a
significant effect on future projects --- either in achievable
precision, detector or beam design or in the running conditions (for
example, neutrino beam vs antineutrino beam).

Besides these oscillation experiments directly addressing the
determination of neutrino parameters, experiments collecting neutrino
interaction data are also of great importance for constraining the models
that are used in oscillation experiments for true-to-visible energy
conversions, predictions of signal and background rates, etc.  MINERvA
is a dedicated neutrino interaction experiment that provides
constraints for oscillation experiments directly and for neutrino
event generators used in neutrino simulations.  The MINERvA
collaboration is currently finishing the analysis of their low-energy
NuMI data (for example,~\cite{Walton:2014esl}) while taking data in
the NuMI medium-energy beam.  A potential upgrade to the MINERvA
detector is being proposed for the intermediate program.  CAPTAIN, a
liquid argon TPC~\cite{Berns:2013usa}, would be combined with the
MINERvA detector to measure neutrino-argon interactions in an energy
range relevant for oscillation physics in ELBNF.

The US-NA61 program is a funded proposal to collaborate with the
NA61/SHINE experiment at CERN~\cite{Gazdzicki:2014bxa} to make hadron
production measurements important for the US neutrino program.  They
will expose targets and replicas of targets used at Fermilab to the
NA61 hadron beam.  Measurements of pion (and other hadron) spectra in
p+C interactions can be used to tune models of primary interactions in
the target and reduce uncertainties on the initial flux in oscillation
experiments.

The NuPRISM collaboration is proposing an experimental method to
remove neutrino interaction uncertainties from oscillation
experiments using water Cherenkov detectors~\cite{Bhadra:2014oma}.  NuPRISM would measure neutrino
interactions over a continuous range of off-axis angles and use these
measurements to provide a direct measurement of the far detector
lepton kinematics for any given set of oscillation parameters.  With
this, they can mostly remove neutrino interaction modeling
uncertainties from oscillation measurements.  This project is being
proposed for the intermediate program.

There is also a group of longer term planned experiments designed to
measure the last unknown elements of the three neutrino mixing model.
These include JUNO, Hyper-K, PINGU, \textsc{Theia}, Daedalus, and ELBNF. R\&D
for these large projects will be an important part of the intermediate
neutrino program.

Within this context, the most relevant issues for these experiments
to seek funding in this intermediate time scale are:
\vspace*{-0.4cm}
\begin{itemize}
\item Discovery potential and/or improved physics reach over other experiments
\vspace*{-0.4cm}
\item Leverage existing resources \vspace*{-0.4cm}
\item Visible and significant US participation \vspace*{-0.4cm}
\item Low technical risk/familiar technology \vspace*{-0.4cm}
\item Potential for each to result in several graduate student theses \vspace*{-0.4cm}
\item US contribution of approximately \$1M each \vspace*{-0.4cm}
\end{itemize}

Of the experiments discussed in the three neutrino mixing working group, there are only a few proposals that will seek funding for the intermediate program.  The above requirements are well fulfilled by these proposals, including proposed experiments
measuring cross sections such as NuPrism and CAPTAIN-MINERvA, as well
as upgrades to existing detectors like gadolinium in Super-K.  In
addition, R\&D for future projects described above is
relevant, in particular for those which can lead to clearly definable US
roles in international collaborations.
