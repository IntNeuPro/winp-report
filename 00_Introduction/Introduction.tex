\section{Introduction}
\label{sec:Introduction}


The US neutrino community gathered at the Workshop on the Intermediate
Neutrino Program (WINP) at Brookhaven National Laboratory February
4--6, 2015 to explore opportunities in neutrino physics over the next
five to ten years. Scientists from particle, astro-particle and
nuclear physics~\footnote{The nuclear physics community is currently
  in the midst of a Long Range Planning process that includes neutrino
  physics.}  participated in the workshop. The US High Energy Physics
community is now launched on the path defined by the 2014 P5 report.
Two of the five P5 Science Drivers motivate neutrino physicists,
\textit{``Pursue the physics associated with neutrino mass''} and
\textit{``Explore the unknown: new particles, interactions, and
  physical principles''}.

Underlying these Drivers is the fact that in spite of tremendous
recent progress in understanding neutrinos, we still lack a complete
picture for the physical behavior and structure of the
neutrino sector. There are many more questions yet to be answered:
What is the neutrino mass ordering? Do neutrinos exhibit the same
matter/antimatter symmetry seen in the charged leptons and quarks
(Dirac fermions) or do they have a completely different structure
(e.g., Majorana fermions)?  Do they violate CP symmetry?  What is the
absolute mass scale of the neutrino sector and is it consistent with
that implied from limits obtained by cosmological observation? Are
experimental hints that there may be additional sterile flavors of
neutrinos valid?  How can we understand neutrino interactions with
nuclei?  Are there non-standard interactions of neutrinos representing
beyond-the-Standard Model physics? 

There are furthermore outstanding questions which can be answered
using neutrinos as a probe, such as: What is the detailed mechanism
behind stellar-collapse supernovae, and what physics can we learn from
a supernova neutrino burst? Is the diffuse flux of neutrinos from past
supernovae consistent with expectations from cosmology?  Where do
ultra high energy neutrinos come from?  What fraction of the Earth's
radiated heat comes from radioactivity?  What fraction of the Sun's
energy comes from the CNO cycle?  What fission isotopes are the main
producers of neutrinos and decay heat in a nuclear reactor core?

In accordance with one specific P5 recommendation, a large international
collaboration (Experimental program at the Long-Baseline Neutrino
Facility, ELBNF) has been formed to perform a long-baseline neutrino
oscillation experiment with an underground liquid argon time
projection chamber and a new neutrino beam from
Fermilab.  The main physics goals of ELBNF are to address some of the
outstanding neutrino questions, by measuring the value of
the CP-violating parameter $\delta$ and providing a definitive
determination of the mass hierarchy, as well as to search for baryon
number violation and record a burst of core-collapse supernova
neutrinos.  The US community is also engaged in development of a
short-baseline neutrino (SBN) program hosted at at Fermilab, to operate
coherently with the long-baseline program in order to address some of the 
short-baseline neutrino anomalies and support R\&D towards ELBNF. 
These activities are consistent with P5 recommendations.

The Fermilab long-baseline effort is exciting and compelling, but
is a long-term effort.  Physics results and technical development in
the short term (this decade) are essential for health of the field,
for motivating young scientists and for sustaining innovation.  The
long-baseline oscillation program addresses some of the most critical
questions in particle physics --- yet other physics questions
associated with neutrinos, including the ones listed above, deserve
attention as well.  Some of these are best addressed with smaller,
lower-cost, dedicated experiments which can be completed
on a shorter time scale than ELBNF.  In addition, some issues will
need large, advanced-technology detectors which require
significant R\&D for their realization.

The P5 report states:
\textit{Some of the biggest scientific questions driving the field can only be
addressed by large and mid-scale experiments. However, small-scale
experiments can also address many of the questions related to the
Drivers. These experiments combine timely physics with opportunities
for a broad exposure to new experimental techniques, provide
leadership roles for young scientists, and allow for partnerships
among universities and national laboratories. In our budget exercises,
we main- tained a small projects portfolio to preserve budgetary space
for a number of these important small projects, whose costs are
typically less than \$20M. These projects individually are not large
enough to come under direct P5 review. Small investments in large,
multi-disciplinary projects, as well as early R\&D for some project
concepts, were also accounted for here.}\\
\textit{{Recommendation 4: Maintain a program of projects of all
  scales, from the largest international projects to mid- and
  small-scale projects.}}


WINP explored opportunities for the broad US neutrino community to
pursue physics in the next five to ten years, including projects of
all scales.  In addition to small- and mid-scale project ideas, WINP
considered (consistent with the P5 recommendation) contributions to
large offshore and multi-disciplinary projects, and to R\&D efforts
aimed towards future large experimental concepts.  Some of these ideas
and efforts are synergistic with the ELBNF and SBN efforts; others are
complementary.  Theory efforts related to the proposed long- and
short-term experimental programs were also discussed. WINP also
covered activities such as searches for neutrinoless double beta decay
and direct mass measurements that are part of the neutrino physics
program in nuclear physics.

The workshop was organized into two sets of parallel working group
sessions, divided by physics topics and technology.  Physics working
groups covered topics on Sterile Neutrino, Neutrino Mixing, Neutrino
Interactions, Neutrino Properties and Astrophysical Neutrinos.
Technology sessions were organized into Theory, Short-Baseline
Accelerator Neutrinos, Reactor Neutrinos, Detector R\&D and Source,
Cyclotron and Meson Decay at Rest sessions.  Each working group
formulated a set of bullet points with key findings and
recommendations, which will be summarized in this report.

In addition to members of the particle and nuclear physics
communities, representatives from the agencies participated in the workshop.

A major area of discussion centered around a possible Department of
Energy Office of High Energy Physics Funding Opportunity Announcement
(FOA) relevant to neutrino physics over a five-year timescale.
Discussion sessions included community suggestions on physics topics
and parameters for such a FOA.  The general outcome of this discussion
was that the community believes that a broader rather than a narrower
scientific scope is important for any such FOA.  There was no clear
consensus on the appropriate fraction of available funds to be
allocated to R\&D efforts, but R\&D was felt to be important.
Potential for publishable results --- either physics, or key technical
results that advance the field --- within five years should be the
most important criterion.  There was also community support to enable
proposals for theoretical effort.  These can be highly cost-effective
and may strongly enhance the success of experimental efforts.
The community felt that a suite of new experiments should be
initiated, consistent with the P5 recommendation, to increase the
breadth of the intermediate program beyond SBN at FNAL. A general
sense emerged that some part of this new program should include
sterile neutrino searches complementary to SBN and that part of this
program should include some of the other exciting new initiatives
discussed.

Although much discussion focused on the potential HEP FOA, the broader
neutrino community was actively engaged at the workshop, and it was
clear that there are additional opportunities for activities on an
intermediate time scale.  For example, another area of discussion was
the importance of determining the nature of the neutrino.  Whether
neutrinos are Majorana or Dirac is of fundamental importance to their
mass generation mechanism and to leptogenesis; the search for
neutrinoless double beta decay offers the most promising (and possibly
only plausible) avenue for addressing this question.  The community
recognizes the importance of the Nuclear Science Advisory Committee
Neutrinoless Double Beta Decay subcommittee for providing guidance on a
strategy for implementation of the next generation neutrinoless double
beta decay experiment.  In addition, many other intermediate
activities --- including R\&D, theory, experiments and contributions
to a variety of international efforts --- are supported by the
National Science Foundation and the DOE Office of Nuclear Physics as
well as the Office of High Energy Physics.
