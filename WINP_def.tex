%------------------------------------------------------------------------------
% New command definitions, constants, etc. for the Sterile Neutrino 
%                               White Paper
% ===========================================================================
%
%  History:
%    PH: 10Oct24: First version
%------------------------------------------------------------------------------
%	Some special symbols etc.:
%       --------------------------
%       -> small numbers
%------------------------------------------------------------------------------
\def\zero{{\scriptscriptstyle 0}}
\def\sone{{\scriptscriptstyle 1}}
\def\stwo{{\scriptscriptstyle 2}}
\def\sthr{{\scriptscriptstyle 3}}
\def\sfou{{\scriptscriptstyle 4}}
\def\sfiv{{\scriptscriptstyle 5}}
\def\sfiv{{\scriptscriptstyle 5}}
\def\ssix{{\scriptscriptstyle 6}}
\def\ssev{{\scriptscriptstyle 7}}
\def\seig{{\scriptscriptstyle 8}}
\def\snin{{\scriptscriptstyle 9}}
\def\sten{{\scriptscriptstyle 10}}
\def\half{{\scriptscriptstyle 1/2}}
%------------------------------------------------------------------------------
%       -> eV and multiples
%------------------------------------------------------------------------------
\def\Ev{\ensuremath{\text{e}\text{V\/}}}
\def\eV{\ensuremath{\text{e}\text{V\/}}}
\def\Kev{\ensuremath{\text{ke}\text{V\/}}}
\def\Mev{\ensuremath{\text{Me}\text{V\/}}}
\def\Gev{\ensuremath{\text{Ge}\text{V\/}}}
\def\GeV{\ensuremath{\text{Ge}\text{V\/}}}
\def\Gevc{\ensuremath{\text{Ge}\text{V\/}/c}}
\def\Gevcs{\ensuremath{\text{Ge}\text{V\/}/c^2}}
\def\Tev{\ensuremath{\text{Te}\text{V\/}}}
\def\ev{\ensuremath{\,\text{e}\text{V\/}}}
\def\kev{\ensuremath{\,\text{ke}\text{V\/}}}
\def\mev{\ensuremath{\,\text{Me}\text{V\/}}}
\def\gev{\ensuremath{\,\text{Ge}\text{V\/}}}
\def\gevc{\ensuremath{\,\text{Ge}\text{V\/}/c}}
\def\gevcs{\ensuremath{\,\text{Ge}\text{V\/}/c^2}}
\def\tev{\ensuremath{\,\text{Te}\text{V\/}}}
%------------------------------------------------------------------------------
%       -> various other units (for math mode only)
%------------------------------------------------------------------------------
\def\pb{\,\text{pb}}
\def\fb{\,\text{fb}}
\def\pbi{\,\text{pb}^{-1}}
\def\m{\,\text{m}}
\def\cm{\,\text{cm}}
\def\mm{\,\text{mm}}
\def\km{\,\text{km}}
\def\mum{\,\mu\text{m}}
\def\mA{\,\text{mA}}
\def\Hz{\,\text{Hz}}
\def\kHz{\,\text{kHz}}
\def\MHz{\,\text{MHz}}
\def\ppm{\,\text{ppm}}
\def\scnd{\,\text{s}}
\def\musec{\,\mu\text{s}}
\def\ns{\,\text{ns}}
\def\rad{\,\text{rad}}
\def\mrad{\,\text{mrad}}
\def\murad{\,\mu\text{rad}}
%------------------------------------------------------------------------------
%       -> commonly used symbols
%------------------------------------------------------------------------------
\def\BR{{\rm BR}}
\def\CC{{\scriptscriptstyle{\rm CC}}}
\def\CI{{\scriptscriptstyle{\rm CI}}}
\def\DA{{\scriptscriptstyle{\rm DA}}}
\def\DO{{D{\O}}\xspace}
\def\DCA{{\rm DCA}}
\def\F{{\cal F}}
\def\GRV{{\scriptscriptstyle{\rm GRV}}}
\def\IC{{\scriptscriptstyle{\rm IC}}}
\def\ISR{{\scriptscriptstyle{\rm ISR}}}
\def\IP{{\rm I$@,$P}}
\def\JB{{\scriptscriptstyle{\rm JB}}}
\def\L{{\scriptscriptstyle{\rm L}}}
\def\LL{{\scriptscriptstyle{\rm LL}}}
\def\LQbar{\overline{\rm LQ}}
\def\LQsub{{\scriptscriptstyle{\rm LQ}}}
\def\LQ{{\rm LQ}}
\def\LR{{\scriptscriptstyle{\rm LR}}}
\def\Ldes{{\cal L}_{\rm des}}
\def\Lumi{{\cal L}}
\def\MC{{\scriptscriptstyle{\rm MC}}}
\def\MSbar{\hbox{$\overline{\rm MS}$}}
\def\NC{{\scriptscriptstyle{\rm NC}}}
\def\NS{{\scriptscriptstyle{\rm NS}}}
\def\Prob{{\cal P}}
\def\QPM{{\scriptscriptstyle{\rm QPM}}}
\def\RL{{\scriptscriptstyle{\rm RL}}}
\def\RPv{{{\not R}_P}}
\def\RR{{\scriptscriptstyle{\rm RR}}}
\def\R{{\scriptscriptstyle{\rm R}}}
\def\SM{{\scriptscriptstyle{\rm SM}}}
\def\SU2U1{{\rm SU}(2)\times{\rm U}(1)}
\def\Ssup{{\scriptscriptstyle{\rm S}}}
\def\XXX{\hbox{\bf ??}}
\def\alsmu#1{{\alpha_s(\mu_{#1}^2)}}
\def\alsmz{{\alpha_s(M_Z^2)}}
\def\alsqs{{\alpha_s(Q^2)}}
\def\als{{\alpha_s}}
\def\ctwb{\cos\theta_{\scriptscriptstyle W}}
\def\ctws{\cos^2\theta_{\scriptscriptstyle W}}
\def\cut{{\rm cut}}
\def\data{{\rm data}}
\def\det{{\rm det}}
\def\dof{{\rm dof}}
\def\elm{{\rm elm}}
\def\epslam{{\epsilon/\Lambda^2}}
\def\exc{{\rm exc}}
\def\exp{{\rm exp}}
\def\gh{{\gamma_h}}
\def\had{{\rm had}}
\def\kg{{k_\gamma}}
\def\lsq#1{{\lambda_{#1}}}
\def\lsqp#1{{\lambda'_{#1}}}
\def\lsqpp#1{{{\lambda''}_{#1}}}
\def\max{{\rm max}}
\def\meas{{\rm meas}}
\def\min{{\rm min}}
\def\obs{{\rm obs}}
\def\ord#1{{\cal O}(#1)}
\def\rnge{{\,\text{--}\,}}
\def\sca{{\rm sca}}
\def\sihat{{\hat\sigma}}
\def\slim{{\rm lim}}
\def\slimb{{\rm lim,B}}
\def\slimr{{\rm lim,R}}
\def\smin{{\rm min}}
\def\smax{{\rm max}}
\def\stwb{\sin\theta_{\scriptscriptstyle W}}
\def\stwss{\sin^4\theta_{\scriptscriptstyle W}}
\def\stws{\sin^2\theta_{\scriptscriptstyle W}}
\def\sys{{\rm sys}}
\def\ta{{\theta^\ast}}
\def\tilR{\tilde R}
\def\tilS{\tilde S}
\def\tilU{\tilde U}
\def\tilV{\tilde V}
\def\tl{{\theta_\ell}}
\def\tot{{\rm tot}}
\def\true{{\rm true}}
\def\tw{\theta_{\scriptscriptstyle W}}
%------------------------------------------------------------------------------
%       -> some math symbols (+,-,...) for usage as mathchar's
%------------------------------------------------------------------------------
\mathchardef\qsm=63
\mathchardef\pls=43
\mathchardef\mns=512
\mathchardef\plm=518
\mathchardef\eql=61
\mathchardef\smallleft=300
\mathchardef\smallright=301
\mathchardef\perslsh=47
\mathchardef\les=316
\mathchardef\gre=318
\mathchardef\leq=532
\mathchardef\grq=533
%------------------------------------------------------------------------------
%       -> characters for typewriter font (used for http addresses)
%------------------------------------------------------------------------------
\chardef\usc=95
\chardef\til=126
%------------------------------------------------------------------------------
%       -> shorthand for "integral limits below and above"
%------------------------------------------------------------------------------
\def\intl{\int\limits}
%------------------------------------------------------------------------------
%       -> footenote markers
%------------------------------------------------------------------------------
\def\fdag{$^{{\displaystyle\dagger})}$}
\def\fddag{$^{{\displaystyle\ddagger})}$}
\def\fstar{$^{{\displaystyle\star})}$}
\def\faste{$^{{\displaystyle\ast})}$}
%------------------------------------------------------------------------------
%       -> d'Alembert operator
%------------------------------------------------------------------------------
\def\sqr#1#2#3{{\vcenter{\hrule height.#3ex\hbox{\vrule width.#2ex height#1ex
    \kern#1ex\vrule width.#3ex}\hrule height.#2ex}}}
\def\dalem{\mathchoice\sqr{1.6}{08}{24}\sqr{1.6}{08}{24}
                      \sqr{1.4}{07}{21}\sqr{1.1}{06}{18}}
%------------------------------------------------------------------------------
%       -> rectangle with arrow ( '-> )
%------------------------------------------------------------------------------
\def\angleto{\vrule width.035em height2.1ex depth-.56ex\unskip\kern-.6ex\to}
%------------------------------------------------------------------------------
%       -> permille sign
%------------------------------------------------------------------------------
\def\perchc#1{{\raise.4ex\hbox{$\mkern4mu#1{\it\perslsh}_
             {\mkern-5mu\scriptscriptstyle{{\rm o}\!{\rm o}}}^
             {\mkern-12.8mu\scriptscriptstyle{\rm o}}$}}}
\def\permil{{\mathchoice\perchc{\displaystyle}\perchc{\textstyle}
             \perchc{\scriptstyle}\perchc{\scriptscriptstyle}}}
%------------------------------------------------------------------------------
%       -> wide bar and bar in parantheses as mathaccents, some applications
%------------------------------------------------------------------------------
\def\widebar#1{\mkern1.5mu\overline{\mkern-1.5mu#1\mkern-1.mu}\mkern1.mu}
\catcode`\@=11 % @ signs are now treated as letters
\def\parenbar{\mathpalette\p@renb@r}
\def\p@renb@r#1#2{\vbox{%
  \ifx#1\scriptscriptstyle \dimen@.7em\dimen@ii.2em\else
  \ifx#1\scriptstyle \dimen@.8em\dimen@ii.25em\else
  \dimen@1em\dimen@ii.4em\fi\fi \offinterlineskip
  \ialign{\hfill##\hfill\cr
    \vbox{\hrule width\dimen@ii}\cr
    \noalign{\vskip-.3ex}%
    \hbox to\dimen@{$\mathchar300\hfil\mathchar301$}\cr
    \noalign{\vskip-.3ex}%
    $#1#2$\cr}}}
\catcode`\@=12 % @ signs are no longer letters
\def\nuan{\ensuremath{\parenbar{\nu}}}
\def\qqbar{\ensuremath{\parenbar{q}}}
\def\pbar{\ensuremath{\widebar{p}}}
\def\qbar{\ensuremath{\widebar{q}}}
\def\dbar{\ensuremath{\widebar{d}}}
\def\ubar{\ensuremath{\widebar{u}}}
\def\sbar{\ensuremath{\widebar{s}}}
\def\cbar{\ensuremath{\widebar{c}}}
\def\bbar{\ensuremath{\widebar{b}}}
\def\tbar{\ensuremath{\widebar{t}}}
\def\nubar{\ensuremath{\widebar{\nu}}}
\def\Dbar{\ensuremath{\widebar{D}}}
\def\ebar{\ensuremath{\widebar{e}}}
%------------------------------------------------------------------------------
%	Some shorthands and utilities:
%       ------------------------------
%       -> vertical rule of width zero and variable height & depth
%------------------------------------------------------------------------------
\newbox\struttbox
\setbox\struttbox=\hbox{\vrule height1.65ex depth.485ex width0pt}
\def\strutt{\relax\ifmmode\copy\struttbox\else\unhcopy\struttbox\fi}
\def\stru#1#2{\relax\ifmmode\hbox{\vrule height#1 depth#2 width0pt}
\else\vrule height#1 depth#2 width0pt\fi}
%------------------------------------------------------------------------------
%       -> underline with resonable distance text - line
%------------------------------------------------------------------------------
\def\uline#1{$\underline{\hbox{#1\strutt}}$}
%------------------------------------------------------------------------------
%       -> roman numbers (uppercase and lowercase)
%------------------------------------------------------------------------------
\def\ronum#1{\uppercase\expandafter{\romannumeral#1}}
\def\ronuml#1{\expandafter{\romannumeral#1}}
%------------------------------------------------------------------------------
%       -> vectors and matrices
%------------------------------------------------------------------------------
\def\vect#1{\begin{matrix}#1\end{matrix}}
\def\pvect#1{\begin{pmatrix}#1\end{pmatrix}}
\def\bvect#1{\begin{bmatrix}#1\end{bmatrix}}
%------------------------------------------------------------------------------
%       -> some alignment tools for tables
%------------------------------------------------------------------------------
\def\cA{{\phantom{0}}}
\def\cB{{\phantom{00}}}
\def\cC{{\phantom{000}}}
\def\cD{{\phantom{0000}}}
\def\cM{{\phantom{\mns}}}
%------------------------------------------------------------------------------
%       -> shorthand for protected citations in captions
%------------------------------------------------------------------------------
\def\pcite#1{\protect\cite{#1}}
%------------------------------------------------------------------------------
%       -> shorthands for equation, figure, table, section references
%------------------------------------------------------------------------------
%\def\eq#1{eq.~(\ref{eq-#1})}
\def\eqsand#1#2{eqs.~(\ref{eq-#1}) and~(\ref{eq-#2})}
\def\eqsto#1#2{eqs.~(\ref{eq-#1}-\ref{eq-#2})}
\def\eqstwo#1#2{eqs.~(\ref{eq-#1},\ref{eq-#2})}
\def\eqsthr#1#2#3{eqs.~(\ref{eq-#1},\ref{eq-#2},\ref{eq-#3})}
\def\eqsfou#1#2#3#4{eqs.~(\ref{eq-#1},\ref{eq-#2},\ref{eq-#3},\ref{eq-#4})}
\def\Eq#1{Equation~(\ref{eq-#1})}
\def\Eqsto#1#2{Equations~(\ref{eq-#1}-\ref{eq-#2})}
%\def\fig#1{Fig.~\ref{fig-#1}}
\def\figs#1{Fig.~\ref{fig-#1}}
\def\Fig#1{Figure~\ref{fig-#1}}
\def\figand#1#2{Figs.~\ref{fig-#1} and~\ref{fig-#2}}
\def\figand#1#2{Figs.~\ref{fig-#1} and~\ref{fig-#2}}
\def\tab#1{Table~\ref{tab-#1}}
\def\tabsand#1#2{Tables~\ref{tab-#1} and~\ref{tab-#2}}
\def\Tab#1{Table~\ref{tab-#1}}
\def\sec#1{Sect.~\ref{sec-#1}}
\def\Sec#1{Section~\ref{sec-#1}}
\def\secand#1#2{Sects.~\ref{sec-#1} and~\ref{sec-#2}}
\def\secsto#1#2{Sects.\ref{sec-#1} to~\ref{sec-#2}}
%------------------------------------------------------------------------------
%	Local laguage changes
%       ---------------------
%------------------------------------------------------------------------------
\def\german#1{\selectlanguage{german}#1\selectlanguage{english}}
%------------------------------------------------------------------------------
%	Redefine mathbf
%       ---------------
%------------------------------------------------------------------------------
\DeclareMathAlphabet{\mathbf}{OT1}{cmr}{bx}{sl}
\def\mb{\mathbf}
%------------------------------------------------------------------------------
%
% New commands:
%
%------------------------------------------------------------------------------
%
\newcommand{\gevsq}     {\mbox{${\rm GeV}^2$}}
\newcommand{\qsd}       {\mbox{${Q^2}$}}
\newcommand{\PTM}       {P_{T\rm miss}}
\newcommand{\x}         {\mbox{${\it x}$}}
\newcommand{\y}         {\mbox{${\it y}$}}
\newcommand{\ye}        {\mbox{${y_{e}}$}}
\newcommand{\smallqsd}  {\mbox{${q^2}$}}
\newcommand{\rambox}    {\mbox{$ \rightarrow $}}
\newcommand{\ra}        {$\rightarrow$}
\newcommand{\ygen}      {\mbox{${y_{gen}}$}}
\newcommand{\yjb}       {\mbox{${y_{_{JB}}}$}}
\newcommand{\yda}       {\mbox{${y_{_{DA}}}$}}
\newcommand{\qda}       {\mbox{${Q^2_{_{DA}}}$}}
\newcommand{\qjb}       {\mbox{${Q^2_{_{JB}}}$}}
\newcommand{\ypt}       {\mbox{${y_{_{PT}}}$}}
\newcommand{\qpt}       {\mbox{${Q^2_{_{PT}}}$}}
\newcommand{\ypr}       {\mbox{${y_{(1)}}$}}
\newcommand{\yprpr}     {\mbox{${y_{(2)}}$}}
\newcommand{\ypps}      {\mbox{${y_{(2)}^2}$}}
\newcommand{\gammah}    {\mbox{$\gamma_{_{H}}$}}
\newcommand{\gammahc}   {\mbox{$\gamma_{_{PT}}$}}
\newcommand{\gap}       {\hspace{0.5cm}}
\newcommand{\gsim}      {\mbox{\raisebox{-0.4ex}{$\;\stackrel{>}{\scriptstyle \sim}\;$}}}
\newcommand{\lsim}      {\mbox{\raisebox{-0.4ex}{$\;\stackrel{<}{\scriptstyle \sim}\;$}}}
\renewcommand{\thefootnote}{\arabic{footnote}}
%
%----------------------------------------------------------------------------
%
% Constants etc.
% ==============
%
%----------------------------------------------------------------------------
% Define (e.g.) working value of theta_23?
%
%----------------------------------------------------------------------------
%
% Paper format
% ============
%
%----------------------------------------------------------------------------
%
%------------------------------------------------------------------------------
%	page layout
%------------------------------------------------------------------------------
\paperheight    29.7cm
\paperwidth     21.0cm
%------------------------------------------------------------------------------
%	... horizontal spacings
%------------------------------------------------------------------------------
\textwidth      16.5cm
\evensidemargin -0.3cm
\oddsidemargin  -0.3cm
%------------------------------------------------------------------------------
%	... vertical spacings
%------------------------------------------------------------------------------
\textheight     675.pt % 650.pt
%\topmargin      -1.5cm -- KL 18Apr06
\footskip        1.0cm
%------------------------------------------------------------------------------
%	interline spacing, paragraph indention etc.
%------------------------------------------------------------------------------
\renewcommand{\baselinestretch}{1.0}
%\renewcommand{\baselinestretch}{1.12}
\renewcommand{\arraystretch}{1.1}
\parindent       0.0pt
\parskip         0.3cm plus0.05cm minus0.05cm
\overfullrule    0.0pt
%------------------------------------------------------------------------------
%	item spacings
%------------------------------------------------------------------------------
%\itemsep    0.2ex plus 0.1ex minus 0.15ex
%\topsep     0.2ex plus 0.1ex minus 0.15ex
%\partopsep  1.2ex plus 0.6ex minus 0.6ex
%\setlength\leftmargini   {1.6em}
%\setlength\leftmarginii  {1.4em}
%\setlength\leftmarginiii {1.2em}
%\setlength\leftmarginiv  {1.0em}
%------------------------------------------------------------------------------
%	"tight item" environments
%------------------------------------------------------------------------------
\newenvironment{tightitemize}[2]
 {\begin{list}{$\bullet$}%
  {\setlength{\leftmargin}{\leftmargini}%
   \setlength{\topsep}{#1}%
   \setlength{\itemsep}{#2}%
  }%
 }%
 {\end{list}}
\newenvironment{tightsubitemize}[2]
 {\begin{list}{$-$}%
  {\setlength{\leftmargin}{\leftmargini}%
   \setlength{\topsep}{#1}%
   \setlength{\itemsep}{#2}%
  }%
 }%
 {\end{list}}
\newenvironment{tightssubitemize}[2]
 {\begin{list}{$\to$}%
  {\setlength{\leftmargin}{\leftmargini}%
   \setlength{\topsep}{#1}%
   \setlength{\itemsep}{#2}%
  }%
 }%
 {\end{list}}
%------------------------------------------------------------------------------
%	float parameters
%------------------------------------------------------------------------------
\renewcommand{\topfraction}{1.}
\renewcommand{\bottomfraction}{1.}
\renewcommand{\textfraction}{0.01}
\setlength{\floatsep}{6pt plus 3pt minus 3pt}
\setlength{\textfloatsep}{12pt plus 8pt minus 4pt}
%------------------------------------------------------------------------------
%	adjust footnote design
%------------------------------------------------------------------------------
\catcode`\@=11 % @ signs are now treated as letters
\newlength{\@fninsert}
\setlength{\@fninsert}{0.6em}
\newlength{\@fnwidth}
\setlength{\@fnwidth}{\textwidth}
\addtolength{\@fnwidth}{-\@fninsert}
\addtolength{\@fnwidth}{-0.4em}
\renewcommand{\@makefntext}[1]%
  {\noindent\makebox[\@fninsert][r]{\@makefnmark}\hfil%
  \parbox[t]{\@fnwidth}{#1}}
\catcode`\@=12 % @ signs are no longer letters
\addtolength{\skip\footins}{2.mm}
%------------------------------------------------------------------------------
%	equation numbers like (n.k): n=section, k=equation
%------------------------------------------------------------------------------
%\numberwithin{equation}{section}
%\renewcommand{\theequation}{\arabic{section}.\arabic{equation}}
%------------------------------------------------------------------------------
%	set section numbering depth (5: allow for sections up to i.j.k.l.m.)
%       and redefine section, subsection, subsubsection, paragraph 
%       to yield a proper heading
%       force four levels of headers in the table of contents
%------------------------------------------------------------------------------
\catcode`\@=11 % @ signs are now treated as letters
\setcounter{secnumdepth}{2}
\setcounter{tocdepth}{2}
\renewcommand\section{\@startsection{section}{1}{\z@}%
                                   {-3.5ex \@plus -1ex \@minus -.2ex}%
                                   {2.3ex \@plus.2ex}%
                                   {\normalfont\Large\bfseries}}
\renewcommand\subsection{\@startsection{subsection}{2}{\z@}%
                                   {-3.25ex\@plus -1ex \@minus -.2ex}%
                                   {1.5ex \@plus .2ex}%
                                   {\normalfont\large\bfseries}}
\renewcommand\subsubsection{\@startsection{subsubsection}{3}{\z@}%
                                   {-3.25ex\@plus -1ex \@minus -.2ex}%
                                   {1.5ex \@plus .2ex}%
                                   {\normalfont\normalsize\bfseries}}
\renewcommand\paragraph{\@startsection{paragraph}{4}{\z@}%
                                   {3.25ex \@plus1ex \@minus.2ex}%
                                   {1.2ex \@plus .2ex}%
                                   {\normalfont\normalsize\bfseries}}
\catcode`\@=12 % @ signs are no longer letters
%------------------------------------------------------------------------------
%	separators for subsections and sections
%------------------------------------------------------------------------------
\def\sectionsep{\strut}
\def\chaptersep{\vfill\eject}
%------------------------------------------------------------------------------
%	Add hyphenation options
%------------------------------------------------------------------------------
\hyphenation{par-ti-cu-lar} \hyphenation{ex-pe-ri-men-tal}
\hyphenation{dif-fe-rent} \hyphenation{bet-we-en}
\hyphenation{mo-du-lus}
%
%--------  End  --------  --------  --------  --------  --------  --------
% Quantum mechanical notation
\newcommand{\bra}[1]{\ensuremath{\langle #1 |}}   % Bra vector
\newcommand{\ket}[1]{\ensuremath{| #1 \rangle}}   % Ket vector
\newcommand{\amp}[3]{\ensuremath{\left\langle #1 \,\left|\, #2%
                     \,\right|\, #3 \right\rangle}}  % QM amplitude
\newcommand{\sprod}[2]{\ensuremath{\left\langle #1 |%
                     #2 \right\rangle}}  % QM scalar product
%\newcommand{\ev}[1]{\ensuremath{\left\langle #1 %
%                     \right\rangle}} % Expectation value
\newcommand{\ds}[1]{\ensuremath{\! \frac{d^3#1}{(2\pi)^3 %
                     \sqrt{2 E_\vec{#1}}} \,}} % Spatial integral
\newcommand{\dst}[1]{\ensuremath{\! %
                     \frac{d^4#1}{(2\pi)^4} \,}} % Space-time integral
\newcommand{\tr}{{\rm tr}} % Trace
\newcommand{\sgn}{{\rm sgn}} % Trace

% Miscellaneous commands
\newcommand{\eps}{\varepsilon}


\newcommand{\eVq}  {\text{eV}^2}
\newcommand{\Dmq}  {\Delta m^2}

%-- symbol shorthands and redefinitions -----------------------------

\newcommand{\bi}{\begin{itemize}}
\newcommand{\ei}{\end{itemize}}
%\newcommand{\ra}{$\rightarrow$}
\newcommand{\be}{\begin{equation}}
\newcommand{\ee}{\end{equation}}
\newcommand{\bea}{\begin{eqnarray}}
\newcommand{\eea}{\end{eqnarray}}
\newcommand{\nn}{\nonumber}
\newcommand{\ldm}{\Delta m_{31}^2}
\newcommand{\sdm}{\Delta m_{21}^2}
\newcommand{\deltacp}{\delta_{\mathrm{CP}}}
\newcommand{\stheta}{\sin^2 2 \theta_{13}}
\newcommand{\deltacpt}{\delta_{\mathrm{CP}}^\mathrm{true}}
\newcommand{\sthetat}{\sin^2 2 \theta_{13}^\mathrm{true}}

\newcommand{\ie}{{\it i.e.}}
\newcommand{\Ie}{{\it I.e.}}
\newcommand{\eg}{{\it e.g.}}
\newcommand{\Eg}{{\it E.g.}}
\newcommand{\cf}{{\it cf.}}
\newcommand{\Cf}{{\it Cf.}}
\newcommand{\etc}{{\it etc.}}
\newcommand{\eq}{Eq.}
\newcommand{\eqs}{Eqs.}
\newcommand{\Def}{Definition}
\newcommand{\fig}{Fig.}
%\newcommand{\Fig}{Fig.}
%\newcommand{\figs}{Figs.}
\newcommand{\Figs}{Figs.}
\newcommand{\Ref}{Ref.}
\newcommand{\Refs}{Refs.}
%\newcommand{\Sec}{Sec.}
\newcommand{\Secs}{Secs.}
\newcommand{\App}{the Appendix}
\newcommand{\Apps}{Appendices}
%\newcommand{\Tab}{Table}
\newcommand{\Tabs}{Tables}

\newcommand{\eet}{\epsilon^m_{e\tau}}
\newcommand{\emt}{\epsilon^m_{\mu\tau}}
\newcommand{\ett}{\epsilon^m_{\tau\tau}}
\newcommand{\eee}{\epsilon^m_{ee}}
\newcommand{\eem}{\epsilon^m_{e\mu}}
\newcommand{\emm}{\epsilon^m_{\mu\mu}}
\newcommand{\eeta}{|\epsilon^m_{e\tau}|}
\newcommand{\emta}{|\epsilon^m_{\mu\tau}|}
\newcommand{\etta}{|\epsilon^m_{\tau\tau}|}
\newcommand{\eeea}{|\epsilon^m_{ee}|}
\newcommand{\eema}{|\epsilon^m_{e\mu}|}
\newcommand{\emma}{|\epsilon^m_{\mu\mu}|}
\newcommand{\eetp}{\phi^m_{e\tau}}
\newcommand{\emtp}{\phi^m_{\mu\tau}}
\newcommand{\eemp}{\phi^m_{e\mu}}

\newcommand{\SuperK}{{\sc Super-Kamiokande}}
\newcommand{\JHFSK}{{\sc T2K}}
\newcommand{\NUMI }{{\sc NO$\nu$A}}
\newcommand{\MINOS}{{\sc MINOS}}
\newcommand{\ICARUS}{{\sc ICARUS}}
\newcommand{\OPERA}{{\sc OPERA}}
\newcommand{\CNGS}{{\sc CNGS}}
\newcommand{\ReactorI}{{\sc Reactor-I}}
\newcommand{\ReactorII}{{\sc Reactor-II}}
\newcommand{\JHFHK}{{\sc T2HK}}
\newcommand{\NuFactII}{{\sc NuFact-II}}

\newcommand{\equ}[1]{\eq~(\ref{equ:#1})}
\newcommand{\figu}[1]{\fig~\ref{fig:#1}}

\newcommand{\as}[2]{\left( \begin{array}{c} #1 \\ #2 \end{array} \right)}

\newcommand{\Slash}[1]{\ooalign{\hfil$ \diagup $\hfil\crcr$#1$}}
