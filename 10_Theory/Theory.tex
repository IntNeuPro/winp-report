
\subsection{Neutrino Theory}
\label{sec:Theory}

During the workshop on Intermediate Neutrino Program held at Brookhaven National Laboratory in February 2015, the neutrino theory community
organized a vibrant session which highlighted the physics goals that can be achieved in the near term future and the role that  theory plays
in achieving these goals.  This section summarizes the consensus that emerged from these discussions. We include specific recommendations which should help the funding agencies in considering support for the neutrino program through investments in theory; including, but not limited to any potential FOA in this area.
%\noindent{\it The role of theory in the Intermediate Neutrino Program:}

Theory plays an essential role in making advances in neutrino physics.
Theory input is necessary in almost all facets of experimental neutrino physics.  Theoretical tools are required
to interpret cross section measurements, to formulate oscillation paradigm with 3 or more neutrinos, to seek
the underlying neutrino mass generation mechanism, to connect experiments with cosmology and astrophysics,
and to infer the fundamental properties of neutrinos.

A relatively small investment in neutrino theory in the intermediate time scale will provide huge dividends that would enrich the
community as a whole.  It would result in timely development of theoretical models for
neutrino-nucleus cross sections, phenomenological studies on
the impact of cross section uncertainties,
%It would result in timely calculations of several needed cross sections,
theoretical cross checks of the 3 neutrino oscillation paradigm with tools such as global fits,
model-building advances to understand large mixings and possible light sterile
neutrinos, and potentially unravel the underlying symmetry that plays a role in neutrino mass generation.
{\it The neutrino theory community thus recommends support for neutrino theory and that the language of any potential FOA for the intermediate neutrino program be broad
enough to allow potential theory proposals.}

%{\it What are the most interesting questions about whose answers we can learn a lot from near-term future experiments (i.e., before ELBNF goes online)?
%}

The theory perspective on the physics goals that can be achieved in the near term future are highlighted below.

\noindent {\bf (A) Fundamental neutrino properties:}

{\bf (1)} {\it Dirac versus Majorana nature of the neutrino:}  Improved searches for neutrinoless double
beta decay are the best bet to address this fundamental question.  If neutrinos obey an inverted mass spectrum
observable signals may be within reach in such experiments in the near future.  Searches for neutrinoless double 
beta decay should continue without waiting for the mass hierarchy measurement, as there may be surprises here, 
such as lepton number violation mediated by TeV scale particles.
{\bf (2)} {\it Direct neutrino mass measurement}: Determining the absolute neutrino mass scale in
tritium beta decay experiments would provide deep insight into the origin of neutrino masses.
{\bf (3)} {\it Sensitivity to mass hierarchy and possibly CP violation:}  Any progress towards measuring the neutrino mass
hierarchy before ELBNF would be highly desirable. In particular, knowing the hierarchy is crucial for the possibility of
obtaining a hint for the value of the leptonic CP violating phase prior to ELBNF.
%In particular, knowing the hierarchy is crucial for
%the future discovery of the leptonic CP violating phase $\delta$.
{\bf (4)}  {\it Consistency checks of the three neutrino oscillation paradigm:}  This requires a variety
experimental information including: (a) Improved knowledge of neutrino oscillation parameters from solar, atmospheric, accelerator and reactor
neutrino experiments; and (b)
Essential information on neutrino interaction rates from experiments, aided by theory.


\noindent {\bf (B) Neutrino interactions:}

Knowing the interaction rates is crucial for addressing many of the important questions in neutrino physics.  For example, 
Over much of the available parameter space, discovery of leptonic CP violation at ELBNF will require, as-yet unachieved, percent-level control over $\nu_e$ appearance signals. A dominant source of uncertainty on this signal is due to the modeling of neutrino interactions with the target nucleus in the near and far detectors. Relating the fundamental quark-level interactions of the neutrino to the complete nuclear response is a difficult field theory problem, involving both particle and nuclear physics.  Unlike in the collider physics community, where there is vibrant interactions between researchers in the domains of (i) detector modeling and event simulation at the hadron level, (ii) perturbative QCD analysis at the parton level and (iii) model building and theoretical interpretation, presently the situation is very different in the analysis of signals at accelerator based neutrino experiments. The analogs of (i) and (ii) above are both relegated to the nuclear physics community. This has two unfortunate outcomes. First, many tools in the particle theorist�s toolkit are not brought to bear on these problems. Second, there is an institutional barrier to communication between those researchers studying neutrino models, and those involved in understanding the experimental analysis of signals and backgrounds.
Collaborative efforts involving HEP theorists, nuclear theorists and neutrino experimenters can uplift this area of research to a level comparable to the one seen in collider physics today.


%We recommend more investment in HEP theory efforts in neutrino cross section calculations, which would help in realistically estimating the %uncertainties involved, and gauging their impact on the experimental neutrino program.


\noindent {\bf (C) Short baseline anomalies and sterile neutrinos:}

{\bf (1)} {\it Understanding anomalies seen in short baseline experiments}:  Unambiguous resolution
in terms of oscillations would require seeing $L/E$ dependence in new/upgraded experiments.
{\bf (2)} {\it Existence of sterile neutrino}:  Discovery of new sterile states in neutrino oscillation
experiments attempting to resolve short baseline anomalies will be foundational.
{\bf (3)} {\it Nonstandard neutrino interactions}:  If discovered, these effects would invalidate the
three neutrino oscillation paradigm, and hint at new physics beyond neutrino masses.


\noindent {\bf (D) Neutrinos in astrophysics and cosmology:}

{\bf (1)}  {\it Supernova neutrinos}:  Anticipating neutrinos from supernova explosions in the near term future, we feel that
investment in understanding the complex dynamics of collective neutrino oscillation is necessary, and that this should happen now,
as it could influence detector design decisions in the next several years.
{\bf (2)} {\it High energy astrophysical neutrinos}: IceCube and its upgrade will tell us more about
the origin of very high energy (PeV scale) astrophysical neutrinos.  Energy spectrum, directional information,
and flavor composition of these events can help us understand the astrophysical sources as well as neutrino properties.
{\bf (3)} {\it Neutrinos in cosmology}: Although indirect, neutrino masses inferred from cosmology
would provide complementary information, and at the same time also test standard cosmological models.

\noindent {\bf (E) Underlying symmetries behind neutrino masses:}

{\bf (1)} {\it Neutrino masses and physics beyond the Standard Model}: What can neutrino experiments teach
us about the underlying symmetries of the theory that generates neutrino masses?  Experiments in the intermediate
time scale can lead to progress in this very basic question.
{\bf (2)} {\it Nucleon decay}:  Ongoing large underground detectors (Super-Kamiokande) which are sensitive to neutrino
oscillation physics continue to be also sensitive to nucleon decay. Discovery of nucleon decay would be monumental.
{\bf (3)} {\it Exotic neutrino properties}:  Information on neutrino properties such as its magnetic moment,
decay lifetime, possible violations of Lorentz invariance and/or CPT invariance, and its interactions with
the dark matter sector would be valuable.  Even if
not found, neutrinos can provide some of the best tests of these fundamental symmetries.

