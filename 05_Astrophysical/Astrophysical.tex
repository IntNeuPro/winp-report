\section{Astrophysical Neutrinos}
\label{sec:Astrophysical}

\subsection{Low Energy Astrophysical Neutrinos}
\label{sec:Astrophysical_low}

Low energy astrophysical neutrinos (typically below about 100 MeV)
result from stars either: a) going about their usual business of
nuclear fusion and emitting a steady stream of solar neutrinos, or b)
living fast, dying young, and leaving a good-looking corpse, in the
process producing a spectacular burst of supernova neutrinos. These
astrophysical neutrinos are important physics messengers, carrying
information on extreme environments and complex processes that would
be otherwise entirely inaccessible, namely what is going on in the
center of stars as they live and die. They have a storied history in
particle physics: the first observations of solar neutrinos (in the
1960's) and supernova neutrinos (in the 1980's) resulted in a shared
2002 Nobel Prize in physics. Famously, the long-standing ``Solar
Neutrino Problem'' turned out to be the first indication of physics
beyond the standard model, a glimpse at the effect of neutrino
oscillations.

What remains to be done on the intermediate time (and money) scale?
For solar neutrinos the main job for the next few years will be to
measure neutrinos produced by the CNO cycle and to explore the MSW
resonance in the sun. For supernova neutrinos from an explosion in our
galaxy, the most important thing is to be ready to collect as much of
this precious data as possible when it arrives, preferably with
complementary technologies to study the various neutrino flavors
and features of the event, particularly the initial neutronization
burst and the final collapse to a black hole. While waiting for the
next galactic burst to arrive the diffuse flux of supernova neutrinos
from ancient supernova explosions can also be collected, provided a
detector is big enough and sensitive enough.

In the coming five years it is expected that an upgraded
Super-Kamiokande enhanced with gadolinium, Borexino, and SNO+ will be
the major solar neutrino experiments in operation; over the same
period Super-Kamiokande with gadolinium, SNO+, and WATCHMAN will be
the main ``new'' projects with significant supernova neutrino
sensitivity. At the same time, CAPTAIN will be providing insight into
the supernova neutrino capabilities of liquid argon detectors like
ELBNE, while R\&D on water-based liquid scintillator should prove a
good investment for longer-term projects like Theia, which would be
capable of studying both solar and supernova neutrinos.

\subsection{High Energy Astrophysical Neutrinos}
\label{sec:Astrophysical_high}

High energy astrophysical neutrinos ($\sim$100~MeV--$10^{20}$~eV) can be
separated into two categories: 
\begin{enumerate}
  \item The ``atmospheric neutrinos'', which are neutrinos produced terrestrially in the earth's
atmosphere as a result of high energy cosmic rays interacting with the
atmosphere, and have been detected with energies of $\sim$100~MeV to over
100~TeV. 
   \item The ``high energy astrophysical neutrinos'', which are
neutrinos produced and originating from extra terrestrial sources, for
example from high energy astrophysical processes in distant
astrophysical objects, high energy cosmic ray interactions in the
universe or neutrino production by exotic matter and particle physics
processes such as dark matter annihilation. The neutrino energies of
these processes are observable above backgrounds at energies of
$\sim$10~TeV and above, and have recently been observed for the first
time at energies up to 2~PeV. This newly discovered neutrino source
holds promise for determining the origin of the highest energy cosmic
rays that has remained a mystery for over 100 years and to open a new
energy window for observing astrophysical objects and the physical
processes responsible for producing the highest energy particles in
the universe.
\end{enumerate}

During this workshop the working group identified three areas
appropriate for possible funding by a program aimed at the
``intermediate'' time and money scale. These include: 
\begin{enumerate}
  \item Funding for a cosmogenic (GZK) neutrino detector using radio detection
techniques. These projects include the ARA and ARRIANA programs both
located in Antarctica.
  \item Funding for R\&D on photodetector,
instrumentation and deployment systems development for large channel
count and physically large volume detectors. Possible projects include
IceCube-Gen2 at South Pole and for related technologies in other
science categories such as PINGU, CHIPS, Theia and others. 
  \item Funding for theoretical work on neutrino production at high energies in
atmospheric neutrinos, and in particular the uncertain flux of the so
called ``prompt'' neutrinos from charm production processes in
cosmic ray interactions with the atmosphere.
\end{enumerate}



