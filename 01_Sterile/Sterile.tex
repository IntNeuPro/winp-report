\subsection{Sterile Neutrinos}
\label{sec:Sterile}

Sterile neutrinos appear in nearly every possible mechanism to explain
neutrino mass. Given that neutrino mass has been the only evidence so
far for physics beyond the Standard Model found in particle physics
(other than astrophysical evidence for dark matter and dark
energy), the quest for understanding the origin of neutrino mass is
one of the priorities of the field.

Apart from these theoretical considerations, there have been
persistent anomalies, observed in electron neutrino appearance and
electron neutrino disappearance (For the disappearance channel
neutrino and antineutrino oscillation probabilities are equal assuming
CPT invariance and therefore we make no distinction in the remainder of
the document), which in combination can be interpreted as evidence
for one or several sterile neutrinos at around a mass of 1~eV and with
oscillation amplitudes as large as 5--10\%. At the same time, the lack
of observations of muon neutrino disappearance at the relevant L/E
values creates a significant tension in global fits. Currently, no
phenomenological models are known that provide a better fit to the
global data than sterile neutrinos. It therefore seems justified to
use sterile neutrino oscillations as a phenomenological proxy for
considering the required experimental capabilities to resolve these
anomalies. In the event that the existence of sterile neutrinos is not
the source of these anomalies, experiments optimized to eV-scale
oscillations can be expected to also have good sensitivity to probe
other new physics scenarios.

During the Snowmass process, P5 deliberations and WINP discussions,
many different experimental approaches to this problem have been
proposed and some have been studied in quite some detail and some have
reached the prototype stage. Sterile neutrino searches build on
detection techniques developed over the last several years, allowing
for efficient experiment deployment at modest cost in some cases.
Among the many proposed experimental approaches, electron neutrino
disappearance searches using radioactive sources and reactors and some
electron neutrino appearance searches seem to fit well within the
scope of an intermediate neutrino program and would complement the
Fermilab short-baseline program. Thus, a modest investment into
sterile neutrino searches has the potential to discover a fundamental 
particle not
predicted in the Standard Model for the first time since the Standard
Model was formulated. This would be a paradigm-shifting discovery
opening up vistas on an entirely new continent of possibilities. The
U.S. community is well poised to play a leadership role and to have a
vibrant program in this field.

In the somewhat longer term, and in particular in anticipation of a
potential discovery of sterile neutrinos, directed R\&D towards novel
detector technologies and neutrino sources also would fit the boundary
conditions of an intermediate neutrino program.

The working group's consensus can be summarized in the following five
recommendations:
\begin{enumerate}
\item{Sterile neutrinos are well motivated in many extensions of the
  Standard Model.  Persistent experimental anomalies have focused
  attention on the eV mass scale.  This makes sterile neutrinos the
  subject of low-risk but potentially high-reward experiments.  
  Therefore, the P5
  Planning Report recommends a targeted set of short-term, small-scale
  experiments.}
\item{Direct tests of existing anomalies should seek to demonstrate
  the sterile neutrino's oscillatory nature via signatures in energy
  and baseline.}
\item{Experiments designed to test both the $\nu_{\mu}$ to $\nu_e$
  appearance and $\nu_e$ disappearance channels are needed.  We must
  ensure that any pion decay beam program has optimized $\nu_{\mu}$
  disappearance sensitivity.}
\item{Below the \$5M level, individual experiments can have an impact on
  oscillation physics results within the WINP time constraints.  In
  the \$5--10M range, several proposed efforts have the potential to
  provide extensive coverage of the suggested oscillation parameter
  space.}
\item{Short-term investment in detector and source R\&D towards future
  sterile oscillation experiments could reduce risk, lead to long-term
  cost savings, and provide the foundation for precision measurements
  in the case of observation of sterile neutrino oscillations.}
\end{enumerate}
